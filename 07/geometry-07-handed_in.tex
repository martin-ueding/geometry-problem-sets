% Copyright © 2015 Martin Ueding <dev@martin-ueding.de>

\documentclass[11pt, english, fleqn, DIV=15, headinclude, BCOR=1cm]{scrartcl}

\usepackage[bibatend]{../header}

\usepackage{tikz}

\usepackage[tikz]{mdframed}
\newmdtheoremenv[%
    backgroundcolor=black!5,
    innertopmargin=\topskip,
    splittopskip=\topskip,
]{theorem}{Theorem}[section]

\usepackage{../my-boxes}

\hypersetup{
    pdftitle=
}

\newcounter{totalpoints}
\newcommand\punkte[1]{#1\addtocounter{totalpoints}{#1}}

\newcounter{problemset}
\setcounter{problemset}{7}

\subject{Geometry in Physics}
\ihead{Geometry in Physics -- Problem Set \arabic{problemset}}

\title{Problem Set \arabic{problemset}}

\publishers{Group 1 -- Jens Boos}
\ofoot{Group 1 -- Jens Boos}

\author{
    Martin Ueding \\ \small{\href{mailto:mu@martin-ueding.de}{mu@martin-ueding.de}}
    \and
    Paul Manz \\ \small{\href{mailto:paul.manz@dreiacht.de}{paul.manz@dreiacht.de}}
}
\ifoot{Martin Ueding, Paul Manz}

\ohead{\rightmark}

\usepackage{multicol}

\renewcommand\thesubsection{\thesection.\alph{subsection}}

\begin{document}

\maketitle

\vspace{3ex}

\begin{center}
    \begin{tabular}{rrr}
        problem & achieved points & possible points \\
        \midrule
        \nameref{homework:1} & & \punkte{24} \\
        \nameref{homework:2} & & \punkte{16} \\
        \nameref{homework:3} & & \punkte{10} \\
        \midrule
        Total & & \arabic{totalpoints}
    \end{tabular}
\end{center}

We will choose coordinates $(x, y, z)$ instead of the generic $(x^1, x^2, x^3)$
since it is less typing and you liked it on sheet~5. When we need to use
indices, we will still use $x^i$ and mean the same coordinates.
Electromagnetism feels to be fundamentally connected to special relativity and
therefore spacetime. We hence use $ct = x^0$ and Greek indices where it eases
notation. This choice of $x^0$ also implies that $\partial_0 = \frac 1c \pd{}t$
by virtue of the chain rule.

\section{Maxwell's field equations}
\label{homework:1}

The $B_i$ seems to be somewhat strange. It suggests that $B$ is a 1-form, which
it is not. We are fine with $B_{ij}$ and $B^k$, where the latter is the Hodge
dual of the 2-form. We will therefore write $B^i$ here.

\subsection{Faraday's lay of induction}

In order so make this shorter, we rewrite the definition of the 2-form
$\alpha$:
\[
    \alpha = E_i \dif x^i \wedge \dif x^0
    + \frac12 \epsilon_{ijk} B^i \dif x^j \wedge \dif x^k.
\]
We have to calculate $\dif \alpha$, which we will do in components. We have:
\[
    [\dif \alpha]_{ij} = \frac{1}{2!} \nabla_{[i} \alpha_{j]}.
\]
Since there is no curvature yet, all the $\Gamma = 0$ and we have $\nabla_i =
\partial_i$. (I know … I just could not resist.) We need to include the
derivative with respect to time. In order to avoid all the $c$ and write it
more compact we will use a Greek index here.
\begin{align*}
    \dif \alpha
    &= \frac12 E_{i,\mu} \dif x^\mu \wedge \dif x^i \wedge \dif x^0
    + \frac14 \epsilon_{ijk} B^i_{,\mu} \dif x^\mu \wedge \dif x^j \wedge \dif x^k
    \intertext{%
        We need to split up those implied sums over $\mu$ into temporal and
        spatial parts in order to group them.
    }
    &= \frac12 E_{i,0} \dif x^0 \wedge \dif x^i \wedge \dif x^0
    + \frac12 E_{i,m} \dif x^m \wedge \dif x^i \wedge \dif x^0
    \\&\qquad
    + \frac14 \epsilon_{ijk} B^i_{,0} \dif x^0 \wedge \dif x^j \wedge \dif x^k
    + \frac14 \epsilon_{ijk} B^i_{,m} \dif x^m \wedge \dif x^j \wedge \dif x^k
    \intertext{%
        Summand 1: There is a form wedged together with itself, this term is
        just zero. Summand 3: We to anticommutations such that the temporal
        form is the very last one.
    }
    &= \frac12 E_{i,m} \dif x^m \wedge \dif x^i \wedge \dif x^0
    + \frac14 \epsilon_{ijk} B^i_{,0} \dif x^j \wedge \dif x^k \wedge \dif x^0
    + \frac14 \epsilon_{ijk} B^i_{,m} \dif x^m \wedge \dif x^j \wedge \dif x^k
    \intertext{%
        Summand 1 \& 2: We factor out the last form and rename the dummy
        indices.
    }
    &= \frac12 \sbr{
        E_{j,i} \dif x^i \wedge \dif x^j
        + \frac12 \epsilon_{lij} B^l_{,0} \dif x^i \wedge \dif x^j
    } \wedge \dif x^0
    + \frac14 \epsilon_{ijk} B^i_{,m} \dif x^m \wedge \dif x^j \wedge \dif x^k
    \intertext{%
        Now we can factor out all the forms. To prepare the next step, we add a
        (now pointless) bracket in the second summand.
    }
    &= \frac12 \sbr{ E_{j,i} + \frac12 \epsilon_{lij} B^l_{,0} }
    \dif x^i \wedge \dif x^j \wedge \dif x^0
    + \frac14 \sbr{\epsilon_{ijk} B^i_{,m}} \dif x^m \wedge \dif x^j \wedge \dif x^k
\end{align*}
When we set this to zero, both brackets have to vanish individually. This will
give us 9 equations for the first and 27 for the second bracket. Since they are
antisymmetrized by the forms, we can add a Levi-Civita symbol to them and
reduce the number of equations to 3 and 1 respectively. The $9 + 27$ equations
are:
\begin{align*}
    E_{j,i} + \frac12 \epsilon_{lij} B^l_{,0} &= 0
    &
    \epsilon_{ijk} B^i_{,m} &= 0 \\
    \intertext{%
        And then we have
    }
    \epsilon^{ijk} E_{j,i} + \frac12 \epsilon^{ijk} \epsilon_{lij} B^l_{,0} &= 0
    &
    \epsilon^{jkm} \epsilon_{ijk} B^i_{,m} &= 0 \\
    \intertext{%
        The first term is the $k$th component of the curl in three dimensions.
        We contract the two Levi-Civita symbols in the second term and in the
        second equation.
    }
    [\curl \vec E]^k + \deltaup^k_l B^l_{,0} &= 0
    &
    2 \deltaup^m_i B^i_{,m} &= 0 \\
    %
    \intertext{%
        Contracting the indices and dropping scalar factors, we get:
    }
    [\curl \vec E]^k + \dot B^k &= 0
    &
    \divergence \vec B &= 0. \\
    \intertext{%
        And we can look at all the three equations at the same time and drop
        the indices.
    }
    \curl \vec E + \dot{\vec B} &= 0
    &
    \divergence \vec B &= 0
\end{align*}
These are the four (or two if you count vector equations as one) homogenious
Maxwell equations. Since those follow from $\dif F = 0$ with $F$ being the
standard field 2-form, we think that we simply have $\alpha = F$ here.

\subsection{Ampère's law and Gauss's law}

We can identify $- H_i \triangleq E_i$ and $D^k \triangleq B^k$ in the above
calculation and directly give $\dif \beta$:
\[
    \dif \beta
    = \sbr{ -H_{j,i} + \frac12 \epsilon_{lij} D^l_{,0} }
    \dif x^i \wedge \dif x^j \wedge \dif x^0
    + \frac12 \epsilon_{ijk} D^i_{,m} \dif x^m \wedge \dif x^j \wedge \dif x^k
\]
We can now add the other form to that.
\begin{align*}
    \dif \beta + 4 \piup \gamma
    &= \sbr{ -H_{j,i} + \frac12 \epsilon_{lij} D^l_{,0} }
    \dif x^i \wedge \dif x^j \wedge \dif x^0
    + \frac12 \epsilon_{ijk} D^i_{,m} \dif x^m \wedge \dif x^j \wedge \dif x^k
    \\&\qquad
    + \frac{4\piup}{c} \epsilon_{ijk} J^k \dif x^i \wedge \dif x^j \wedge \dif x^0
    - 4 \piup \varrho \dif x \wedge \dif y \wedge \dif z
    \intertext{%
        We regroup this by the forms again. For the $D^i_{,m}$ term we use
        perform the antisymmetrization by hand by adding another Levi-Civita
        symbol.
    }
    &= \sbr{ -H_{j,i} + \frac12 \epsilon_{lij} D^l_{,0} + \frac{4\piup}{c}
\epsilon_{ijk} J^k }
    \dif x^i \wedge \dif x^j \wedge \dif x^0
    \\&\qquad
    + \sbr{\frac12 \epsilon^{mjk} \epsilon_{ijk} D^i_{,m} - 4 \piup \varrho} \dif x \wedge \dif y
    \wedge \dif z
    \intertext{%
        We simplify the second term, it gives another divergence like before.
    }
    &= \sbr{ -H_{j,i} + \frac12 \epsilon_{lij} D^l_{,0} + \frac{4\piup}{c}
\epsilon_{ijk} J^k }
    \dif x^i \wedge \dif x^j \wedge \dif x^0
    \\&\qquad
    + \sbr{\divergence \vec D - 4 \piup \varrho} \dif x \wedge \dif y
    \wedge \dif z
\end{align*}
One equation then is $\divergence \vec D = 4 \piup \varrho$. The other one (or
rather, nine) equations has to be simplified according to the same procedure as
before.
\begin{align*}
    - H_{j,i} + \frac12 \epsilon_{lij} D^l_{,0} + \frac{4\piup}{c} \epsilon_{ijk} J^k &= 0
    \intertext{%
        Since those nine equations they are antisymmetric in $i$ and $j$ we can
        rewrite them as three unique equations, just like before.
    }
    - \epsilon^{ijm} H_{j,i} + \frac12 \epsilon^{ijm} \epsilon_{lij} D^l_{,0} +
    \frac{4\piup}{c} \epsilon^{ijm} \epsilon_{ijk} J^k &= 0 \\
    - \epsilon^{ijm} H_{j,i} + \deltaup^m_l D^l_{,0} +
    \frac{8\piup}{c} \deltaup^m_k J^k &= 0 \\
    - [\curl \vec H]^m + D^m_{,0} + \frac{8\piup}{c} J^m &= 0 \\
    - \curl \vec H + \frac 1c \dot{\vec D} + \frac{8\piup}{c} \vec J &= 0
\end{align*}

This can be rewritten in the usual form and gives:
\[
    \curl \vec H - \frac 1c \dot{\vec D} = \frac{8\piup}{c} \vec J
\]
That is the correct result in Gaussian units.

\subsection{Continuity equation}

The equation $\dif b + 4 \piup \gamma = 0$ gives a physical result. Applying
the exterior derivative again on it gives up $\dif \gamma = 0$.

Using the techniques used in the previous problems we can also use the
definition of $\gamma$ and derive that
\[
    \dif \gamma = \divergence \vec J + \dot\varrho.
\]
Setting this to zero gives the known continuity equation which tells us that
charge is conserved, especially over time. $\int_\Sigma \dif \gamma$ then gives
the amount of charge gained or lost in a time interval $\dif t$.

\subsection{Relation of gauge freedom to Poincaré's lemma}

You have shown in the exercise class on 2015-05-18 that the difference between
two local vector potentials on a non-contractable space (like the sphere) can
be rewritten as a gauge transformation of the form $A$ (we just try to avoid to
write “\emph{vector} potential 1-\emph{form}”).

% TODO

\section{Cohomology groups of a torus}
\label{homework:2}

\subsection{Zeroth cohomology group}

We are only looking at 0-forms here, plain functions. If they are closed, this
means that
\[
    \dif f = f_{,i} \dif x^i = 0.
\]
Scalar fields where all partial derivatives vanish have only one basis, $f(x,
y) = 1$. They are not particularly fancy, just constant fields. These forms a
vector space over $\R$, such that $Z^0(M) = \R$. Since there are no $-1$-forms,
$B^0(M) = \varnothing$ as given in the hint.

We are not sure what it means to form the quotient set with the empty set as
the divisor. It probably is the same as dividing through the set that only
contains the identity, $\{ e \}$. If that is the case, we have
\[
    H^0(M) = \frac{Z^p(M)}{\varnothing} = \R.
\]

\subsection{Integrals}

\subsection{Closed forms}

Since the two dimensional torus is a two dimensional space, there are no
3-forms. Every top dimensional forms are closed in this sense.

\section{Homotopy type of a punctured plane}
\label{homework:3}




\end{document}

% vim: spell spelllang=en tw=79
