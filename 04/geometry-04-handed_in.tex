% Copyright © 2015 Martin Ueding <dev@martin-ueding.de>

\documentclass[11pt, english, fleqn, DIV=15, headinclude, BCOR=1cm]{scrartcl}

\usepackage[bibatend, color]{../header}

\usepackage{tikz}

\usepackage[tikz]{mdframed}
\newmdtheoremenv[%
    backgroundcolor=black!5,
    innertopmargin=\topskip,
    splittopskip=\topskip,
]{theorem}{Theorem}[section]

\hypersetup{
    pdftitle=
}

\newcounter{totalpoints}
\newcommand\punkte[1]{#1\addtocounter{totalpoints}{#1}}

\newcounter{problemset}
\setcounter{problemset}{4}

\subject{Geometry in Physics}
\ihead{Geometry in Physics -- Problem Set \arabic{problemset}}

\title{Problem Set \arabic{problemset}}

\publishers{Group 1 -- Jens Boos}
\ofoot{Group 1 -- Jens Boos}

\author{
    Martin Ueding \\ \small{\href{mailto:mu@martin-ueding.de}{mu@martin-ueding.de}}
    \and
    Paul Manz \\ \small{\href{mailto:paul.manz@dreiacht.de}{paul.manz@dreiacht.de}}
}
\ifoot{Martin Ueding, Paul Manz}

\ohead{\rightmark}

\usepackage{multicol}

\renewcommand\thesubsection{\thesection.\alph{subsection}}

\begin{document}

\maketitle

\vspace{3ex}

\begin{center}
    \begin{tabular}{rrr}
        problem & achieved points & possible points \\
        \midrule
        \nameref{homework:1} & & \punkte{10} \\
        \nameref{homework:2} & & \punkte{10} \\
        \nameref{homework:3} & & \punkte{10} \\
        \nameref{homework:4} & & \punkte{20} \\
        \midrule
        Total & & \arabic{totalpoints}
    \end{tabular}
\end{center}

\section{Visualization of differential forms}
\label{homework:1}

\subsection{Visualization}

The forms can be visualized in the dual space. In the dual space, there should
not be a problem by visualizing them with vectors. Vectors are translationally
invariant, so we can draw them anywhere, just like it is done on the problem
set. All four forms are shown in Figure~\ref{fig:forms}.

\begin{figure}[htbp]
    \centering
    \begin{tikzpicture}[scale=2]
        \draw[->] (-.3, 0) -- (2, 0) node[right] {$\dif x^1$};
        \draw[->] (0, -.3) -- (0, 2) node[above] {$\dif x^1$};

        \draw (1, .1) -- (1, -.1) node[below] {$1$};
        \draw (.1, 1) -- (-.1, 1) node[left] {$1$};

        \draw[very thick, ->] (2.2, .4) -- ++(1, 0) node[midway, sloped, above] {$\dif x^1$};
        \draw[very thick, ->] (3, 1) -- ++(0, 1) node[midway, sloped, above] {$\dif x^2$};

        \draw[very thick, ->] (0, 0) -- ++(30:2) node[midway, sloped, above] {$\dif r$};
        \draw[very thick, ->] (30:2) -- ++(120:.5) node[midway, sloped, above] {$\dif \phi$};
    \end{tikzpicture}
    \caption{%
        Visualization of $\dif x^1$, $\dif x^2$, $\dif r$ and $\dif \phi$ in
        the dual space.
    }
    \label{fig:forms}
\end{figure}

\paragraph{Transformation of $\dif r$ into Cartesian coordinates}

The forms given in polar coordinates have to be transformed into Cartesian
coordinates to draw them into the Cartesian plot. Although the representation
of $\dif r$ in Cartesian coordinates seems “obvious”, we would like to engage
the full machinery of pullbacks and pushforwards in order to see whether we
understood the concepts correctly.

The transformation which brings us from Cartesian to polar coordinates is given
by
\begin{align*}
    \varphi \colon &\R^2_\text{Cartesian} \to \R^2_\text{polar} \\
                &x \ev_x + y \ev_y \mapsto \sqrt{x^2 + y^2} \, \ev_r + \arccos\del{\frac xr}
    \ev_\phi,
\end{align*}
although it is not well defined for all angles $\phi$. As long as we restrict
this to the first quadrant, everything is fine. Let us restrict this to the
open set $(0, \infty) \times (0, \infty)$ to make it easier. Then $\varphi$ is
a diffeomorphism.

Now we use a pullback which will let us bring tensors the other way around.
This is best expressed more formally:
\[
    \varphi^* \colon T_{\varphi(\xi)}\R^2_\text{polar} \to
    T_{\xi}\R^2_\text{Cartesian}
\]
$\xi$ is a point given in Cartesian coordinates. The components of a tensor
$\tens\omega$ then transform like the following:
\[
    [\varphi^* \omega]^{\nu_1 \ldots}{}_{\mu_1\ldots}(\xi) =
    \omega^{\alpha_1 \ldots}{}_{\beta_1\ldots} (\varphi(\xi))
    \pd{y^{\beta_1}}{x^{\mu_1}} \cdots
    \pd{x^{\nu_1}}{y^{\alpha_1}} \cdots
    \eqnsep
    \vec y := \vec\varphi(\xi).
\]

With our given 1-form, this collapses to a single lower index:
\begin{align*}
    [\varphi^* \omega]_i(\xi)
    &= \omega_j(\varphi(\xi)) \pd{\varphi^j}{x^i}(\xi).
    \intertext{%
        $\omega$ is just $\dif r$, so we have the components $\omega_r = 1$ and
        $\omega_\phi = 0$. $r$ is not an index which takes on numbers, but we
        rather want it to be interpreted that $j \in \{r, \phi\}$. This is a
        bad overloading of symbols, but we hope that it is somewhat
        decipherable. Since there is only one component in $\omega$ the
        equation simplifies to
    }
    &= 1 \cdot \pd{\varphi^r}{x^i}(\xi).
    \intertext{%
        Using the explicit form of the map $\vec \varphi$, the partial
        derivatives are
    }
    &= \sbr{\frac{x_i}{\sqrt{x^2 + y^2}}}_{\xi}.
\end{align*}
This is now expressed in terms of Cartesian coordinates. It simplifies
to use the polar coordinates to write this more compact. Then we have
\[
    [\varphi^* \dif r]_i \dif x^i
    = \dif r
    = \cos(\phi) \dif x + \sin(\phi) \dif y.
\]

\paragraph{Conversion of $\dif \phi$}

Using the same derivation we can derive an expression for $\dif \phi$:
\begin{align*}
    [\varphi^* \dif \phi]_i(\xi)
    &= \dif \phi_j(\varphi(\xi)) \pd{\varphi^j}{x^i}(\xi) \\
    &= 1 \cdot \pd{\varphi^\phi}{x^i}(\xi)
\end{align*}
The derivatives are given by:
\[
    \pd{\varphi^\phi}{x} = - \frac{y}{r^2} = - \frac{\sin(\phi)}r
    \eqnsep
    \pd{\varphi^\phi}{y} = \frac{x}{r^2} = \frac{\cos(\phi)}r
\]
Therefore
\[
    \dif \phi = - \frac{\sin(\phi)}r \, \dif x + \frac{\cos(\phi)}r \, \dif y.
\]

\subsection{Values of 1-forms on vectors}

\paragraph{Values}

The vectors given in the image are:
\[
    \vec v_1 = \begin{pmatrix} 0 \\ 1 \end{pmatrix}
    \eqnsep
    \vec v_2 = \begin{pmatrix} -1 \\ -1 \end{pmatrix}
    \eqnsep
    \vec v_3 = \begin{pmatrix} 1 \\ -1 \end{pmatrix}
\]

Then we can compute:
\begin{gather*}
    \omega_1(\vec v_1) = \dif x^1 (\partial_2) = 0 \\
    \omega_1(\vec v_2) = \dif x^1 (-\partial_1-\partial_2) = -1 \\
    \omega_1(\vec v_3) = \dif x^1 (\partial_1-\partial_2) = 1 \\
    \omega_2(\vec v_1) = x^1 \, \dif x^2 (\partial_2) = x^1 \\
    \omega_2(\vec v_2) = x^1 \, \dif x^2 (-\partial_1-\partial_2) = -x^1 \\
    \omega_2(\vec v_3) = x^1 \, \dif x^2 (\partial_1-\partial_2) = - x^1  \\
    \omega_3(\vec v_1) = \dif r (-\partial_1-\partial_2) = \sin(\phi) \\
    \omega_3(\vec v_2) = \dif r (\partial_1-\partial_2) = -\cos(\phi) - \sin(\phi) \\
    \omega_3(\vec v_3) = \dif r (\partial_1-\partial_2) = \cos(\phi) - \sin(\phi)
\end{gather*}

Everything is summarized in Table~\ref{tab:values}.

\begin{table}[htbp]
    \centering
    \begin{tabular}{c|ccc}
        & $\vec v_1$ & $\vec v_2$ & $\vec v_3$ \\
        \midrule
        $\omega_1$ & $0$ & $-1$ & $1$ \\
        $\omega_2$ & $x^1$ & $- x^1$ & $- x^1$ \\
        $\omega_3$ & $\sin(\phi)$ & $-\cos(\phi) - \sin(\phi)$ & $\cos(\phi) - \sin(\phi)$ \\
    \end{tabular}
    \caption{%
        Summary of value of the 1-forms on the vectors.
    }
    \label{tab:values}
\end{table}

\paragraph{Geometric interpretation}

1-forms are densities in one dimension. Examples are the force $\vec F$ or the
electric field $\vec E$. So the value $\omega(\vec v)$ then gives the amount of
“work” done along the tangential vector $\vec v$ against the “force” $\omega$.

If one uses the metric and computes the vector which are dual to the 1-forms,
this is a scalar product. But we think that this is exactly what is \emph{not}
wanted here.

\subsection{Values of 2-forms on parallelograms}

For your convenience, all the results are aggregated in
Table~\ref{tab:values2}. The derivations are following, of course.

\begin{table}[htbp]
    \centering
    \begin{tabular}{c|ccc}
        & $(\vec v_1, \vec w_1)$ & $(\vec v_2, \vec w_2)$ & $(\vec v_3, \vec w_3)$ \\
        \midrule
        $\eta_1$ & $1$ & $2$ & $-1$ \\
        $\eta_2$ & $x^1+x^2$ & $2\sbr{x^1+x^2}$ & $-\sbr{x^1+x^2}$ \\
        $\eta_3$ & $1$ & $2$ & $-1$
    \end{tabular}
    \caption{%
        Summary of value of the 2-forms on the vector pairs.
    }
    \label{tab:values2}
\end{table}

\paragraph{First form}

The first value is given by
\begin{align*}
    \eta_1(\vec v_1, \vec w_1)
    &= \sbr{\dif x^1 \wedge \dif x^2}(\vec v_1, \vec w_1) \\
    &= \sbr{\dif x^1 \wedge \dif x^2} \del{
        \begin{pmatrix}
            1 \\ 0
        \end{pmatrix},
        \begin{pmatrix}
            1 \\ 1
        \end{pmatrix}
    } \\
    &= \sbr{\dif x^1 \wedge \dif x^2} \del{\ev_1, \ev_1 + \ev_2} \\
    &= \sbr{\dif x^1 \wedge \dif x^2} \del{\ev_1, \ev_1}
    + \sbr{\dif x^1 \wedge \dif x^2} \del{\ev_1, \ev_2}. \\
    \intertext{%
        The first part is zero. The differential form must be antisymmetric in
        its arguments. The second summand is 1.
    }
    &= 1.
\end{align*}

The second one:
\begin{align*}
    \eta_1(\vec v_2, \vec w_2)
    &= \sbr{\dif x^1 \wedge \dif x^2} \del{\ev_1 - \ev_2, 2 \ev_2} \\
    &= 2 \sbr{\dif x^1 \wedge \dif x^2} \del{\ev_1, \ev_2}
    - 2 \sbr{\dif x^1 \wedge \dif x^2} \del{\ev_2, \ev_2}
    \intertext{%
        Here the first term is one, the second term is zero.
    }
    &= 2
\end{align*}

And the last one with $\eta_1$.
\begin{align*}
    \eta_1(\vec v_3, \vec w_3)
    &= \sbr{\dif x^1 \wedge \dif x^2} \del{- \ev_1, \ev_2} \\
    &= - \sbr{\dif x^1 \wedge \dif x^2} \del{\ev_1, \ev_2} \\
    &= - 1
\end{align*}

\paragraph{Second form}

For 2-forms, there is just one basis element, namely $\dif x^1 \wedge \dif
x^2$. So all 2-forms are just scalar densities and cannot depend on the
direction. 2-forms are also Hodge dual to scalars, which also do not depend on
the direction. Therefore, all those 2-forms are scalar multiples of each other.
This makes it really easy to compute the remaining values.

First we have to rewrite $\eta_2$ in terms of this \emph{one} basis value. We
think that this form of $\eta_2$ is just given this way to encourage people to
think before doing a lot of calculations. The wedge product is antisymmetric,
so this can be written more compact as
\[
    \eta_2
    = \sbr{x^1 \, \dif x^1 \wedge \dif x^2 - x^2 \, \dif x^2 \wedge \dif x^1}
    = \sbr{ x^1 + x^2} \sbr{\dif x^1 \wedge \dif x^2}
    = \sbr{ x^1 + x^2} \eta_1.
\]
Therefore, the values are all the same, just multiplied with that factor.

\paragraph{Third form}

The third form is a 2-form as well, so it must be a scalar multiple of $\dif
x^1 \wedge \dif x^2$.
\begin{align*}
    \eta_3
    &= r \, \dif r \wedge \dif \phi
    \intertext{%
        Now we use the expressions in the Cartesian dual basis that we have
        derived previously.
    }
    &= r \sbr{\cos(\phi) \dif x + \sin(\phi) \dif y} \wedge
    \sbr{-\frac{\sin(\theta)}r \dif x + \frac{\cos(\phi)}r \dif y}
    \intertext{%
        We would get four terms in the multiplication. There are only two
        nonzero ones since the wedge product is antisymmetric and the first and
        last terms would cancel then. We only keep the inner and outer terms.
    }
    &= \cos(\phi)^2 \dif x \wedge \dif y - \sin(\phi)^2 \dif y \wedge
    \dif x \\
    \intertext{%
        Here we use the antisymmetry of the wedge product.
    }
    &= \cos(\phi)^2 \dif x \wedge \dif y + \sin(\phi)^2 \dif x \wedge
    \dif y \\
    \intertext{%
        Now we can factor out the basis element of the 2-forms.
    }
    &= \sbr{\cos(\phi)^2 + \sin(\phi)^2} \dif x \wedge \dif y \\
    &= \dif x \wedge \dif y
\end{align*}
So this is exactly $\eta_1$ and since the 2-forms cannot depend on direction in
$\R^2$, the values when acting on the vector pairs are exactly the as for
$\eta_1$.

\section{Straight forward calculations}
\label{homework:2}

\section{Hyperbolid coordinates}
\label{homework:3}


\section{Pullback}
\label{homework:4}



\end{document}

% vim: spell spelllang=en tw=79
