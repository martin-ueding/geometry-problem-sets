% Copyright © 2015 Martin Ueding <dev@martin-ueding.de>

\documentclass[11pt, english, fleqn, DIV=15, headinclude, BCOR=1cm]{scrartcl}

\usepackage[bibatend, color]{../header}

\usepackage{tikz}

\usepackage[tikz]{mdframed}
\newmdtheoremenv[%
    backgroundcolor=black!5,
    innertopmargin=\topskip,
    splittopskip=\topskip,
]{theorem}{Theorem}[section]

\hypersetup{
    pdftitle=
}

\newcounter{totalpoints}
\newcommand\punkte[1]{#1\addtocounter{totalpoints}{#1}}

\newcounter{problemset}
\setcounter{problemset}{2}

\subject{Geometry in Physics}
\ihead{Geometry in Physics -- Problem Set \arabic{problemset}}

\title{Problem Set \arabic{problemset}}

\publishers{Group 1 -- Jens Boos}
\ofoot{Group 1 -- Jens Boos}

\author{
    Martin Ueding \\ \small{\href{mailto:mu@martin-ueding.de}{mu@martin-ueding.de}}
    \and
    Paul Manz \\ \small{\href{mailto:paul.manz@dreiacht.de}{paul.manz@dreiacht.de}}
}
\ifoot{Martin Ueding, Paul Manz}

\ohead{\rightmark}

\usepackage{multicol}

\renewcommand\thesubsection{\thesection.\alph{subsection}}

\begin{document}

\maketitle

\vspace{3ex}

\begin{center}
    \begin{tabular}{rrr}
        problem number & achieved points & possible points \\
        \midrule
        1 & & \punkte{20} \\
        2 & & \punkte{10} \\
        2 & & \punkte{20} \\
        \midrule
        Total & & \arabic{totalpoints}
    \end{tabular}
\end{center}

\section{Linear algebra}

\subsection{Dual vector space}

The dual space $V^*$ is defined as the space of all mappings from the vector
space $V$ to the underlying field, here the reals $\R$:
\[
    V^* = \operatorname{hom}(V, \R).
\]

For $V^*$ to be a vector space, it has to be a group under “$+$” and a group
under “$\cdot$” (except for zero) as well has have two laws of distributivity.
The space of linear maps is a group under “$+$” since it has a neutral element
that maps everything to zero, there is an additive inverse and the addition is
associative. The scalar multiplication is a group on the set except for the
zero vector. Since all the maps are linear, they fulfill linearity and
therefore distributivity.

\subsection{Dual basis}

A basis is defined as a set of linear independent vectors that span the
complete space. The basis $\{\ev_i\}$ of $V$ is already given, so we know that
they are linearly independent and that they generate all of $V$. The dual space
$V^*$ has the same dimension as $V$ since every dimension in $V$ means another
degree of freedom in the mappings $V\to\R$. The basis of $V^*$ is given with
the $\{\ev^i\}$ which fulfill $\ev^i(\ev_j) = \deltaup^i_j$.

We now have to show that this dual basis consists of linear independent
elements and that it spans the whole space. Since we only have $n$ of them,
they are linear independent if they span the whole $n$-dimensional space.

\subsection{Change of basis}

\subsection{Isomorphism}

\subsection{Dual dual space}

\section{Inner product}

\section{Isomorphy of $S^2$}


\end{document}

% vim: spell spelllang=en tw=79
