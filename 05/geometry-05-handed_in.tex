% Copyright © 2015 Martin Ueding <dev@martin-ueding.de>

\documentclass[11pt, english, fleqn, DIV=15, headinclude, BCOR=1cm]{scrartcl}

\usepackage[bibatend, color]{../header}

\usepackage{tikz}

\usepackage[tikz]{mdframed}
\newmdtheoremenv[%
    backgroundcolor=black!5,
    innertopmargin=\topskip,
    splittopskip=\topskip,
]{theorem}{Theorem}[section]

\usepackage{../my-boxes}

\hypersetup{
    pdftitle=
}

\newcounter{totalpoints}
\newcommand\punkte[1]{#1\addtocounter{totalpoints}{#1}}

\newcounter{problemset}
\setcounter{problemset}{5}

\subject{Geometry in Physics}
\ihead{Geometry in Physics -- Problem Set \arabic{problemset}}

\title{Problem Set \arabic{problemset}}

\publishers{Group 1 -- Jens Boos}
\ofoot{Group 1 -- Jens Boos}

\author{
    Martin Ueding \\ \small{\href{mailto:mu@martin-ueding.de}{mu@martin-ueding.de}}
    \and
    Paul Manz \\ \small{\href{mailto:paul.manz@dreiacht.de}{paul.manz@dreiacht.de}}
}
\ifoot{Martin Ueding, Paul Manz}

\ohead{\rightmark}

\usepackage{multicol}

\renewcommand\thesubsection{\thesection.\alph{subsection}}

\begin{document}

\maketitle

\vspace{3ex}

\begin{center}
    \begin{tabular}{rrr}
        problem & achieved points & possible points \\
        \midrule
        \nameref{homework:1} & & \punkte{20} \\
        \nameref{homework:2} & & \punkte{10} \\
        \nameref{homework:3} & & \punkte{20} \\
        \midrule
        Total & & \arabic{totalpoints}
    \end{tabular}
\end{center}

\section{Winding form}
\label{homework:1}

We will use $x := x^1$ and $y := x^2$ to ease the notation of squared.

\subsection{Closed}

We have
\[
    \omega := \frac{1}{x^2 + y^2} \sbr{x \dif y - y \dif x}.
\]
Computation of the exterior derivative does not involve any new principles.
First the product rule:
\begin{align*}
    \dif \omega
    &= \dif \sbr{\frac{1}{x^2 + y^2}} \wedge \sbr{x \dif y - y \dif x}
    + \frac{1}{x^2 + y^2} \dif \sbr{x \dif y - y \dif x} \\
    \intertext{%
        We use the chain rule next:
    }
    &= -\frac{1}{[x^2 + y^2]^2} [2 x \dif x + 2 y \dif y] \wedge \sbr{x \dif y - y \dif x}
    + \frac{1}{x^2 + y^2} [\dif x \wedge \dif y - \dif y \wedge \dif x] \\
    \intertext{%
        We expand the brackets in the first summand and use the antisymmetry in
        the second.
    }
    &= -\frac{1}{[x^2 + y^2]^2} [2 x^2 \dif x \wedge \dif y - 2 y^2 \dif y
    \wedge \dif x]
    + \frac{2}{x^2 + y^2} \dif x \wedge \dif y \\
    \intertext{%
        Using the antisymmetry in the first term, we can factor out $[x^2 +
        y^2]$.
    }
    &= -\frac{2}{x^2 + y^2} \dif x \wedge \dif y + \frac{2}{x^2 + y^2} \dif x \wedge \dif y \\
    &= 0
\end{align*}
So $\omega$ is a closed form.

\subsection{Pullback}

$\omega$ is defined on $U$. It is a 1-form. On $\gamma$ it still is a 1-form,
but the basis there is just $\dif t$.
The pullback is generally given by
\begin{align*}
    [\gamma^*\omega](\tau) \dif t
    &= \omega_i(\gamma(\tau)) \pd{\gamma^i}{t}(\tau) \dif t.
    \intertext{%
        Since the domain of $\gamma$ only has one dimension, there is no
        summation over $t$ as there would be in case of multiple basis
        elements. The partial derivatives of $\gamma$ are:
        \[
            \dot{\vec \gamma}(t) = n
            \begin{pmatrix}
                - \sin(nt) \\ \cos(nt)
            \end{pmatrix}.
        \]
        We therefore have
    }
    &= \sbr{
        \frac{1}{x^2+y^2} \sbr{y n \sin(nt) + x n \cos(nt)} \dif t
    }_{\vec x = \vec\gamma(\tau)} \\
    &= \sbr{
        \frac{n}{x^2+y^2} \sbr{x \cos(nt) + y \sin(nt)} \dif t
    }_{\vec x = \vec\gamma(\tau)} \\
    \intertext{%
        We now evaluate this form at the position $\vec\gamma(t)$.
    }
    &= \frac{n}{\cos(nt)^2+\sin(nt)^2} \sbr{\cos(nt)^2 + \sin(nt)^2} \dif t \\
    &= n \dif t
\end{align*}

\subsection{Integral}

The integral is given by
\[
    \intop_\gamma \omega
    = \intop_{(0, 2\piup)} \gamma^*\omega
    = n \intop_{(0, 2\piup)} \dif t
    \overset{\text{FT}}= n \intop_{\partial(0, 2\piup)} t
    = n [- 0 + 2\piup]
    = 2 n \piup,
\]
where we have used the fundamental theorem of exterior calculus for the third
equality.

\subsection{Not exact}

If $\omega$ was exact, then there would be an $\alpha$ such that $\omega = \dif
\alpha$ would hold globally. If that was the case, the integral would have
vanished since a closed curve has no boundary.
\[
    \intop_\gamma \dif \alpha \overset{\text{FT}}= \intop_{\partial \gamma}
    \alpha = 0
\]
This curve $\gamma$ can also be thought as the boundary of the unit disk. The
boundary of a boundary is zero, so $\partial^2 = 0$ ($\partial$ is nilpotent),
just like $\dif^2 = 0$. This is a hint at the duality of integration domains
and forms.

The integral over a closed curve (around a hole in the domain) does not vanish.
Therefore, the form $\omega$ cannot be exact.

\section{Pullback and exterior derivative}
\label{homework:2}

The pullback of the exterior derivative is given by:
\[
    [F^* \dif \omega]_{ij\ldots l}(x) = [\dif \omega]_{ab\ldots d} (F(x))
    \pd{F^a}{x^i}
    \pd{F^b}{x^j}
    \ldots
    \pd{F^d}{x^l}
    .
\]
The other direction requires a few more steps. We first apply the exterior
derivative onto the pullback and add another index since we go from a p-form to
a $(p+1)$-form.
\begin{align*}
    \sbr{\dif [F^*\omega]_{j\ldots l}}_{i}(x)
    &= \dif \sbr{\omega_{b\ldots d} (F(x))
        \pd{F^b}{x^j}
        \ldots
        \pd{F^d}{x^l}
    }_i
    \intertext{%
        Now we can apply the chain rule to the $\omega$ and get another
        $\partial F/\partial x$ out which needs to be summed over.
    }
    &= [\dif \omega]_{ab\ldots d} (F(x)) \pd{F^a}{x^i} \pd{F^b}{x^j} \ldots
    \pd{F^d}{x^l}
\end{align*}
The exterior derivatives of the partial derivatives of $F$ vanish since the
partial derivatives commute whereas the exterior derivative is antisymmetric.

This show that the pullback and the exterior derivative commute.

\section{Poincaré lemma}
\label{homework:3}

\subsection{Two pullbacks}

\paragraph{Monomials}

First we want to define the monomials that we use. Let $\{i_n\}$ be an index
set with $p$ elements. Then we can define a $p$-form which does not contain
$\dif t$:
\[
    \omega := a\del{\{x^i\}} \bigwedge_{n = 1}^p \dif x^{i_n}.
\]
Similarly, we can define a $p$-form that does contain $\dif t$. Here we index
set must have one element less in order to still give a $p$-form.
\[
    \psi := a\del{t, \{x^i\}} \dif t \wedge \bigwedge_{n = 1}^{p-1} \dif x^{i_n}
\]
We can also use the index 0 to denote the time as usual in SR and GR.

Using both together, we can unify those definition into a third one which uses
an index set $\{ \nu_n \}$ which also contains the zero index.
\[
    \phi := a\del{\{x^\mu\}} \bigwedge_{n = 1}^{p} \dif x^{\nu_n}
\]

\paragraph{Pullback}

For both forms, we can write the same pullback. The case $\omega$ is just the
special case that all $\phi_{\ldots0\ldots} = 0$, i.e.\ there is no $\dif t$
component in $\phi$.

We write the pullback in its general form here:
\begin{align*}
    \sbr{j_T^* \phi}_{i\ldots k}(\vec x)
    &= \phi\del{j_T(T, \vec x)}_{\alpha\ldots \gamma}
    \pd{j_T^\alpha}{x^i}(\vec x)
    \ldots
    \pd{j_T^\gamma}{x^k}(\vec x).
    \intertext{%
        We can inline $j_T$ directly since it is just this simple expression.
    }
    &= \phi(T, \vec x)_{\alpha\ldots \gamma}
    \pd{j_T^\alpha}{x^i}(\vec x)
    \ldots
    \pd{j_T^\gamma}{x^k}(\vec x).
    \intertext{%
        If one of $\alpha\ldots\gamma$ is 0, we have $j_T^0(\vec x) = T$.
        Those functions are constant with respect to any $x^i$, so those
        derivatives vanish. This means that we can go back to Latin indices for
        the contracted ones.
    }
    &= \phi(T, \vec x)_{a \ldots c}
    \pd{j_T^a}{x^i}(\vec x)
    \ldots
    \pd{j_T^c}{x^k}(\vec x).
    \intertext{%
        Now there are only Latin indices and it condenses nicely to
    }
    &= \phi(T, \vec x)_{a \ldots c}
    \deltaup^a_i \ldots \deltaup^c_k.
    \intertext{%
        Contracting the now unneeded indices, we have the final form:
    }
    \sbr{j_T^* \phi}_{i\ldots k}(\vec x)
    &= \phi(T, \vec x)_{i \ldots k}.
\end{align*}

Using this general form, we know that $\omega$ transform as one would expect,
and $j_T^* \psi = 0$ since the $\dif t$ component of $\phi$ did not get summed
over.

In summary: $j_T^*$ either sets the time to $T$ when $\phi$ does not contain
$\dif t$, else it is mapped to zero. This also makes sense if $\dif t$ is
thought of as a density in time. Considering a density to a fixed time point
(zero measure), the value of that form there will be zero.

\subsection{Linear map}

\paragraph{Exterior derivatives}

The exterior derivative of the general $\phi$ is
\[
    \dif \phi = a_{,\lambda} \del{\{x^\mu\}} \dif x^\lambda \wedge \bigwedge_{n = 1}^{p} \dif x^{\nu_n}.
\]
Not all summands in $\sum_\lambda$ contribute. In the case $\lambda \in
\{\nu_n\}$ the antisymmetry of the wedge product will set this term to zero.

There are no terms with $a$ (without derivatives) since the application of the
exterior derivative on any of the forms gives zero, $\dif^2 = 0$.

\begin{question}
    One writes $\dif^2 = 0$ but also $\int \dif^3x \, f(\vec x)$. How does that
    fit together?
\end{question}

There is an additional remark we can make for $\omega$ and $\psi$ since we know
that do not and do contain a $\dif t = \dif x^0$ term. In $\omega$, we can be
sure there will not be another $\dif t$ term and therefore we can write it like
so:
\[
    \dif \omega = \sbr{a_{,0}(\tens x) \dif t +  a_{,l}(\tens x) \dif x^l} \wedge
    \bigwedge_{n = 1}^{p} \dif x^{i_n}.
\]

For $\psi$ we know that there is a $\dif t$ term and that this will cancel with
the $\dif t$ we get with the time derivative of $a$. There is a $\dif t$ from
the other derivatives, however. We have the $\dif t$ in the front of the other
basis forms without loss of generality, one could have it anywhere and
introduce a constant number of minus signs.
\[
    \dif \psi = a_{,l}(\tens x) \dif x^l \wedge \dif t \wedge
    \bigwedge_{n = 1}^{p-1} \dif x^{i_n}
\]

\paragraph{Case without $\dif t$}

We have seen in the previous subsection that these kinds of forms, $\omega$,
transform in the pullback as usual. The linear mapping $K$ acts on those with
$K(\omega) = 0$. This also tells us that $\dif(K \omega) = 0$ as well.

$\dif \omega$ consists of two parts, one that contains $\dif t$ and one which
does not. The application of $K$ to it looks like this:
\begin{align*}
    K(\dif \omega)
    &= K\del{\sbr{a_{,0}(\tens x) \dif t + a_{,l}(\tens x) \dif x^l} \wedge
    \bigwedge_{n = 1}^{p} \dif x^{i_n}} \\
    \intertext{%
        $K$ looks linear, so the part without $\dif t$ will give zero.
    }
    &= K\del{a_{,0}(\tens x) \dif t \wedge \bigwedge_{n = 1}^{p} \dif x^{i_n}}
    \intertext{%
        Now we insert the definition of $K$.
    }
    &= \sbr{\intop_{[0,1]} a_{,0}(\tens x) \dif t} \bigwedge_{n = 1}^{p} \dif x^{i_n}
    \intertext{%
        We could regard $a$ as being defined on $I$ only and then use the
        fundamental theorem of exterior calculus and rewrite this as
    }
    &= \sbr{\intop_{\partial[0,1]} a(\tens x)} \bigwedge_{n = 1}^{p} \dif x^{i_n}
    \intertext{%
        which is fancy for
    }
    &= \sbr{a(1, \vec x) - a(0, \vec x)} \bigwedge_{n = 1}^{p} \dif x^{i_n}
    \intertext{%
        Since the pullback with $j_T^*$ is rather simple, we can write this as
    }
    &= j_1^*(\omega) - j_0^*(\omega).
\end{align*}
Recalling that $K(\omega) = 0$, the Equation~(9) on the problem set is shown
for the case that the form does not contain $\dif t$.

\paragraph{Case with $\dif t$}

Now we look at this again for the form $\psi$ which contains $\dif t$. We know
that the pullback of this will be zero. So the right hand side of Equation~(9)
will be zero already. Now we have to show that the left hand side is also zero.
We begin with the first part.
\begin{align*}
    K(\dif \psi)
    &= K\del{a_{,l}(\tens x) \dif x^l \wedge \dif t \wedge
    \bigwedge_{n = 1}^{p-1} \dif x^{i_n}}
    \intertext{%
        In order to apply the definition, we need to move the $\dif t$ to the
        front, this introduces a minus sign. Using the alleged linearity of
        $K$, we pull this out.
    }
    &= - K\del{a_{,l}(\tens x) \dif t \wedge \dif x^l \wedge
    \bigwedge_{n = 1}^{p-1} \dif x^{i_n}}
    \intertext{%
        Now we can apply the definition. In order to perform the integration we
        write $\tens x = t \oplus \vec x$ as two arguments.
    }
    &= - \sbr{\int_0^1 a_{,l}(t, \vec x) \dif t} \dif x^l \wedge
    \bigwedge_{n = 1}^{p-1} \dif x^{i_n}
    \intertext{%
        Since we do not know anything about $a$, we will just denote the
        indefinite integral by $A$ and assume that it a reasonable function
        such that we can interchange integration and differentiation.
    }
    &= \sbr{A_{,l}(0, \vec x) - A_{,l}(1, \vec x)} \dif x^l \wedge
    \bigwedge_{n = 1}^{p-1} \dif x^{i_n}
\end{align*}

Now we can look at the second part.
\begin{align*}
    K(\psi)
    &= K\del{a(t, \vec x) \dif t \wedge \bigwedge_{n = 1}^{p-1} \dif x^{i_n}}
    \intertext{%
        This is straightforward to apply since the $\dif t$ is right up front.
    }
    &= \sbr{A(1, \vec x) - A(0, \vec x)} \bigwedge_{n = 1}^{p-1} \dif x^{i_n}
    \intertext{%
        Now we take the exterior derivative of both sides.
    }
    \dif \del{K(\psi)}
    &= \dif \del{\sbr{A(1, \vec x) - A(0, \vec x)} \bigwedge_{n = 1}^{p-1}
    \dif x^{i_n}}
    \intertext{%
        Due to the nilpotency of the exterior derivative, the only non-zero
        terms are the one where the derivative acts on the $A$.
    }
    &= \sbr{\dif A(1, \vec x) - \dif A(0, \vec x)} \bigwedge_{n = 1}^{p-1}
    \dif x^{i_n}
    \intertext{%
        We can write the exterior derivative of $A$ in components:
    }
    &= \sbr{A_{,l}(1, \vec x) - A_{,l}(0, \vec x)} \dif x^l \wedge \bigwedge_{n = 1}^{p-1}
    \dif x^{i_n}
\end{align*}
This is the exact negative of the previous term. Summed together, they give
zero, just like was required for the left hand side.

\subsection{Forms on contractible set}

\begin{question}
    When I took a GR lecture, the requirement for the Poincaré lemma was that
    the domain of the closed form is star shaped. The example $\R^2 \setminus
    \{ \vec 0 \}$ considered in the problem “\nameref{homework:1}” is an
    example of a domain which is not star shaped. Therefore it is not
    surprising that the closed form is not exact.

    The first impression I had from the definition was that it simply meant
    that every point is path-connected to $x_0$. So what I read is
    \[
        \exists x_0 \in U \colon
        \forall x \in U \colon
        \exists \phi \colon I \times U \to U \colon
        \phi(1, x) = x \land \phi(0, x) = x_0.
    \]
    Reading more closely I noticed the actual order of the quantifiers:
    \[
        \exists x_0 \in U \colon
        \exists \phi \colon I \times U \to U \colon
        \forall x \in U \colon
        \phi(1, x) = x \land \phi(0, x) = x_0.
    \]

    Since there is \emph{a single} map $\phi$ that has to work for all the
    points simultaneously, this restricts the topology of the set a lot more,
    right? I tried to come up with a map to show that $\R^2 \setminus \{ \vec 0
    \}$ is contractible to a point to see whether this definition excludes one
    example of domain which is is not star shaped. My mapping is the innocent
    looking
    \[
        \phi(t, x) = [1-t] \, x + t x_0.
    \]
    This fulfils the boundary conditions and it is smooth since it is just a
    linear interpolation between the two points. If I take $\vec x_0 = (1,
    0)^\text T$ then I will have the problem that although the boundary
    conditions are met, the image of $\phi(0.5, U)$ will contain the point
    $\vec 0$ which is excluded from $U$. So I have $\phi(0.5, (-1, 0)^\text T)
    = \vec 0$ which is against the definition of the domain of the map $\phi$.
    Is that the point where my counterexample breaks and the $\R^2 \setminus \{
    \vec 0 \}$ is not contractible to a point?
\end{question}

\end{document}

% vim: spell spelllang=en tw=79
