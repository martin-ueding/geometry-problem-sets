% Copyright © 2015 Martin Ueding <dev@martin-ueding.de>

\documentclass[11pt, english, fleqn, DIV=15, headinclude, BCOR=1cm]{scrartcl}

\usepackage[bibatend, color]{../header}

\usepackage{tikz}

\usepackage[tikz]{mdframed}
\newmdtheoremenv[%
    backgroundcolor=black!5,
    innertopmargin=\topskip,
    splittopskip=\topskip,
]{theorem}{Theorem}[section]

\hypersetup{
    pdftitle=
}

\newcounter{totalpoints}
\newcommand\punkte[1]{#1\addtocounter{totalpoints}{#1}}

\newcounter{problemset}
\setcounter{problemset}{5}

\subject{Geometry in Physics}
\ihead{Geometry in Physics -- Problem Set \arabic{problemset}}

\title{Problem Set \arabic{problemset}}

\publishers{Group 1 -- Jens Boos}
\ofoot{Group 1 -- Jens Boos}

\author{
    Martin Ueding \\ \small{\href{mailto:mu@martin-ueding.de}{mu@martin-ueding.de}}
    \and
    Paul Manz \\ \small{\href{mailto:paul.manz@dreiacht.de}{paul.manz@dreiacht.de}}
}
\ifoot{Martin Ueding, Paul Manz}

\ohead{\rightmark}

\usepackage{multicol}

\renewcommand\thesubsection{\thesection.\alph{subsection}}

\begin{document}

\maketitle

\vspace{3ex}

\begin{center}
    \begin{tabular}{rrr}
        problem & achieved points & possible points \\
        \midrule
        \nameref{homework:1} & & \punkte{20} \\
        \nameref{homework:2} & & \punkte{10} \\
        \nameref{homework:3} & & \punkte{20} \\
        \midrule
        Total & & \arabic{totalpoints}
    \end{tabular}
\end{center}

\section{Winding form}
\label{homework:1}

We will use $x := x^1$ and $y := x^2$ to ease the notation of squared.

\subsection{Closed}

We have
\[
    \omega := \frac{1}{x^2 + y^2} \sbr{x \dif y - y \dif x}.
\]
Computation of the exterior derivative does not involve any new principles.
First the product rule:
\begin{align*}
    \dif \omega
    &= \dif \sbr{\frac{1}{x^2 + y^2}} \wedge \sbr{x \dif y - y \dif x}
    + \frac{1}{x^2 + y^2} \dif \sbr{x \dif y - y \dif x} \\
    \intertext{%
        We use the chain rule next:
    }
    &= -\frac{1}{[x^2 + y^2]^2} [2 x \dif x + 2 y \dif y] \wedge \sbr{x \dif y - y \dif x}
    + \frac{1}{x^2 + y^2} [\dif x \wedge \dif y - \dif y \wedge \dif x] \\
    \intertext{%
        We expand the brackets in the first summand and use the antisymmetry in
        the second.
    }
    &= -\frac{1}{[x^2 + y^2]^2} [2 x^2 \dif x \wedge \dif y - 2 y^2 \dif y
    \wedge \dif x]
    + \frac{2}{x^2 + y^2} \dif x \wedge \dif y \\
    \intertext{%
        Using the antisymmetry in the first term, we can factor out $[x^2 +
        y^2]$.
    }
    &= -\frac{2}{x^2 + y^2} \dif x \wedge \dif y + \frac{2}{x^2 + y^2} \dif x \wedge \dif y \\
    &= 0
\end{align*}
So $\omega$ is a closed form.

\subsection{Pullback}

$\omega$ is defined on $U$. It is a 1-form. On $\gamma$ it still is a 1-form,
but the basis there is just $\dif t$.
The pullback is generally given by
\begin{align*}
    [\gamma^*\omega](\tau) \dif t
    &= \omega_i(\gamma(\tau)) \pd{\gamma^i}{t}(\tau) \dif t.
    \intertext{%
        Since the domain of $\gamma$ only has one dimension, there is no
        summation over $t$ as there would be in case of multiple basis
        elements. The partial derivatives of $\gamma$ are:
        \[
            \dot{\vec \gamma}(t) = n
            \begin{pmatrix}
                - \sin(nt) \\ \cos(nt)
            \end{pmatrix}.
        \]
        We therefore have
    }
    &= \sbr{
        \frac{1}{x^2+y^2} \sbr{y n \sin(nt) + x n \cos(nt)} \dif t
    }_{\vec x = \vec\gamma(\tau)} \\
    &= \sbr{
        \frac{n}{x^2+y^2} \sbr{x \cos(nt) + y \sin(nt)} \dif t
    }_{\vec x = \vec\gamma(\tau)} \\
    \intertext{%
        We now evaluate this form at the position $\vec\gamma(t)$.
    }
    &= \frac{n}{\cos(nt)^2+\sin(nt)^2} \sbr{\cos(nt)^2 + \sin(nt)^2} \dif t \\
    &= n \dif t
\end{align*}

\subsection{Integral}

The integral is given by
\[
    \intop_\gamma \omega
    = \intop_{(0, 2\piup)} \gamma^*\omega
    = n \intop_{(0, 2\piup)} \dif t
    \overset{\text{FT}}= n \intop_{\partial(0, 2\piup)} t
    = n [- 0 + 2\piup]
    = 2 n \piup,
\]
where we have used the fundamental theorem of exterior calculus for the third
equality.

\subsection{Not exact}

If $\omega$ was exact, then there would be an $\alpha$ such that $\omega = \dif
\alpha$ would hold globally. If that was the case, the integral would have
vanished since a closed curve has no boundary.
\[
    \intop_\gamma \dif \alpha \overset{\text{FT}}= \intop_{\partial \gamma}
    \alpha = 0
\]
This curve $\gamma$ can also be thought as the boundary of the unit disk. The
boundary of a boundary is zero, so $\partial^2 = 0$ ($\partial$ is nilpotent),
just like $\dif^2 = 0$. This is a hint at the duality of integration domains
and forms.

The integral over a closed curve (around a hole in the domain) does not vanish.
Therefore, the form $\omega$ cannot be exact.

\section{Pullback and exterior derivative}
\label{homework:2}

The pullback of the exterior derivative is given by:
\[
    [F^* \dif \omega]_{ij\ldots l}(x) = [\dif \omega]_{ab\ldots d} (F(x))
    \pd{F^a}{x^i}
    \pd{F^b}{x^j}
    \ldots
    \pd{F^d}{x^l}
    .
\]
The other direction requires a few more steps. We first apply the exterior
derivative onto the pullback and add another index since we go from a p-form to
a $(p+1)$-form.
\begin{align*}
    \sbr{\dif [F^*\omega]_{j\ldots l}}_{i}(x)
    &= \dif \sbr{\omega_{b\ldots d} (F(x))
        \pd{F^b}{x^j}
        \ldots
        \pd{F^d}{x^l}
    }_i
    \intertext{%
        Now we can apply the chain rule to the $\omega$ and get another
        $\partial F/\partial x$ out which needs to be summed over.
    }
    &= [\dif \omega]_{ab\ldots d} (F(x)) \pd{F^a}{x^i} \pd{F^b}{x^j} \ldots
    \pd{F^d}{x^l}
\end{align*}
The exterior derivatives of the partial derivatives of $F$ vanish since the
partial derivatives commute whereas the exterior derivative is antisymmetric.

\section{Poincaré lemma}

\label{homework:3}

\end{document}

% vim: spell spelllang=en tw=79
