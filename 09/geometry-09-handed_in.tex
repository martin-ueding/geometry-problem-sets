% Copyright © 2015 Martin Ueding <dev@martin-ueding.de>

\documentclass[11pt, english, fleqn, DIV=15, headinclude, BCOR=1cm]{scrartcl}

\usepackage[bibatend]{../header}

\usepackage{tikz}

\usepackage[tikz]{mdframed}
\newmdtheoremenv[%
    backgroundcolor=black!5,
    innertopmargin=\topskip,
    splittopskip=\topskip,
]{theorem}{Theorem}[section]

\usepackage{../my-boxes}

\hypersetup{
    pdftitle=
}

\newcounter{totalpoints}
\newcommand\punkte[1]{#1\addtocounter{totalpoints}{#1}}

\newcounter{problemset}
\setcounter{problemset}{9}

\subject{Geometry in Physics}
\ihead{Geometry in Physics -- Problem Set \arabic{problemset}}

\title{Problem Set \arabic{problemset}}

\publishers{Group 1 -- Jens Boos}
\ofoot{Group 1 -- Jens Boos}

\author{
    Martin Ueding \\ \small{\href{mailto:mu@martin-ueding.de}{mu@martin-ueding.de}}
    \and
    Paul Manz \\ \small{\href{mailto:paul.manz@dreiacht.de}{paul.manz@dreiacht.de}}
}
\ifoot{Martin Ueding, Paul Manz}

\ohead{\rightmark}

\usepackage{multicol}

\renewcommand\thesubsection{\thesection.\alph{subsection}}

\begin{document}

\maketitle

\vspace{3ex}

\begin{center}
    \begin{tabular}{rrr}
        problem & achieved points & possible points \\
        \midrule
        \nameref{homework:1} & & \punkte{15} \\
        \nameref{homework:2} & & \punkte{15} \\
        \nameref{homework:3} & & \punkte{10} \\
        \nameref{homework:4} & & \punkte{10} \\
        \midrule
        Total & & \arabic{totalpoints}
    \end{tabular}
\end{center}

\section{Hodge star operator}
\label{homework:1}

The trick is to choose $\beta$ to stamp out the components of $\alpha$ which
are wanted. One can restrict those components to be from $1, \ldots, p$ and
then later generalize this to all indices. We will use the index set $\{ i_k
\}$ which contains all the indices between 1 and $n$. We chose
\[
    \beta = \bigwedge_{k=1}^p \dif x^{i_k},
\]
so all the components are one if $k \leq p$, zero otherwise. Since the
equations are linear in $\beta$, we could also add other components
$\beta_{i_k}$ here. Using this choice of $\beta$, we can write the scalar
product like so, using another name for the index set with $j_k := i_k$ such
that the Einstein summation convention still works:
\begin{align*}
    \bracket{\beta, \alpha}
    &= \frac{1}{p!} \Bracket{\bigwedge_{k=1}^p \dif x^{i_k}, \bigwedge_{l=1}^p \dif x^{j_l}}
    \alpha_{j_1 \ldots j_p} \omega.
    \intertext{%
        The lecture notes then give the definition of such a scalar product.
        The indices $k$ and $l$ are not really defined, but there are curly
        braces used. So we think that we are supposed to build a matrix where
        each component has this scalar product as its value.
    }
    &= \frac{1}{p!} \det\del{\set{\bracket{\dif x^{i_k}, \dif x^{j_l}}}_{kl}}
    \alpha_{j_1 \ldots j_p} \omega
    \intertext{%
        The scalar product of two simple 1-forms is (or defines?) the metric
        tensor $\tens g$.
    }
    &= \frac{1}{p!} \det\del{\set{g^{i_k j_l}}_{kl}} \alpha_{j_1 \ldots j_p}
    \omega
    \intertext{%
        Next we can apply the formula of the determinant using the Levi-Civita
        symbol. The formula given in the lecture notes has the Levi-Civita
        symbol like $\epsilonup^{i_1 \ldots i_p}_{l_1 \ldots l_p}$. This
        Levi-Civita symbol has upper and lower indices which are atop of each
        other. This does not make too much sense. The way it is written there,
        the metric tensor would just raise the indices of the Levi-Civita
        symbol giving $\epsilonup^{i_1 \ldots i_p \, j_1 \ldots j_p}$, which
        will then create the dual of the form, but with respect to $2p$ and not
        $n$ dimensions. Looking at the definition of the determinant by
        \textcite[§13.2]{penrose-road_to_reality} it becomes clear that there
        are \emph{two} Levi-Civita symbols. So the one from the lecture notes
        might be some sort of double-Levi-Civita symbol? This seems to be the
        generalized Kronecker $\delta$ judging from its properties
        \parencite{wikipedia/Kronecker_delta}. The problem set also has this
        Levi-Civita symbol with upper and lower indices atop of each other, but
        there it means $\epsilonup^{i_1 \ldots i_p}{}_{i_{p+1} \ldots i_n}$.
        Using the two distinct symbols, we get the following:
    }
    &= \frac{1}{p!} \epsilonup^{i_1 \ldots i_p}
    \sbr{\prod_{m = 1}^p g^{l_m j_m}} \epsilonup_{l_1 \ldots l_p}
    \alpha_{j_1 \ldots j_p} \omega.
    \intertext{%
        Then we can raise the indices on the second symbol using all the metric
        tensors.
    }
    &= \frac{1}{p!} \epsilonup^{i_1 \ldots i_p} \epsilonup^{j_1 \ldots j_p}
    \alpha_{j_1 \ldots j_p} \omega
    \intertext{%
        Remember that the index names are all the same, $i_k = j_k$. They just
        have different names such that the summation convention still works
        out. This means that the Levi-Civita symbols do not really add
        anything. For any unique chose of indices, both will be either zero or
        $\pm 1$. Taking the square of this will either be zero or one. The
        components of the $p$-form $\alpha$ are zero if there is an index
        occurring twice. So the Levi-Civita symbols can just be dropped. Then
        there is no summation convention and we can just rename the indices.
        The $1/p!$ factor came from this summation, so we drop that as well.
    }
    &= \alpha^{i_1 \ldots i_p} \omega
\end{align*}

Now we need to approach this equation from the other side. We choose $\beta$
the same way.
\begin{align*}
    \beta \wedge [\hodge \alpha]
    &= \sbr{\bigwedge_{k = 1}^p \dif x^{i_k}} \wedge [\hodge \alpha]
    \intertext{%
        Only the components of $\hodge \alpha$ that do not contain any of the
        $\{ i_k \}$ contribute to this. We can therefore use all the indices
        $\{ i_l | l > p \}$ for $\hodge \alpha$. The summation will give us a
        normalization factor.
    }
    &= \frac{1}{[n - p]!} \sbr{\bigwedge_{k = 1}^p \dif x^{i_k}} \wedge
    \sbr{\bigwedge_{l = p+1}^n \dif x^{j_l}} [\hodge \alpha]_{j_{p+1} \ldots
    j_n}
    \intertext{%
        We can join the wedging and get closer to the volume form $\omega$.
    }
    &= \frac{1}{[n - p]!} \sbr{\bigwedge_{k = 1}^n \dif x^{i_k}} [\hodge
    \alpha]_{i_{p+1} \ldots i_n}
    \intertext{%
        The big wedge is now the volume form divided by $\sqrt{|\tens g|}$.
        Since the volume form contains the basis elements in order, we do not
        need the normalization factor any more.
    }
    &= \frac{\omega}{\sqrt{|\tens g|}} [\hodge
    \alpha]_{i_{p+1} \ldots i_n}
\end{align*}
Since  $\alpha$ and  $\hodge \alpha$ are completely antisymmetric and we only
use the last $n - p$ indices, we can use a Levi-Civita symbol to connect the
form to its dual by equating both sides of the definition and translate between
the first $p$ and last $n-p$ indices.
\begin{align*}
    \frac{\omega}{\sqrt{|\tens g|}} [\hodge \alpha]_{i_{p+1} \ldots i_n}
    &= \frac{1}{p!} \epsilonup_{i_1 \ldots i_p i_{p+1} \ldots i_n} \alpha^{i_1 \ldots i_p} \omega
    \intertext{%
        We can cancel $\omega$ to get an index only expression and move the
        other parts to one side to isolate an expression for the components of
        the Hodge dual.
    }
    [\hodge \alpha]_{i_{p+1} \ldots i_n}
    &= \frac{\sqrt{|\tens g|}}{p!}
    \epsilonup_{i_1 \ldots i_p i_{p+1} \ldots i_n} \alpha^{i_1 \ldots i_p}
    \intertext{%
        This is the definition given in the lecture notes. The problem set has
        its the other way around, it does not give the components of the Hodge
        dual, but it shows how specific components transform under the dual.
        All is needed is to exchange the dual with the form itself and adjust
        the number of indices and normalization factors. Also cancel the
        components of $\alpha$ since they are not needed in the definition.
    }
    \hodge \bigwedge_{k = 1}^p \dif x^{i_k}
    &= \frac{\sqrt{|\tens g|}}{[n-1]!}
    \epsilonup_{i_1 \ldots i_p i_{p+1} \ldots i_n} \bigwedge_{k = p+1}^n \dif x^{i_k}
\end{align*}

\section{Coderivative}
\label{homework:2}

The first hint given on the problem set looks awkward. The product sign should
be scoped to the metric tensor only and now including the Levi-Civita symbol
that itself contains an index $k$. As stated, the hint would be:
\begin{align*}
    \epsilonup_{i_1 \ldots i_n}
    &:= \prod_{k=1}^n g^{i_k j_k} \epsilonup_{j_1 \ldots j_k} \\
    &=
    g^{i_1 j_1} \epsilonup_{j_1}
    g^{i_2 j_2} \epsilonup_{j_1 j_2}
    g^{i_3 j_3} \epsilonup_{j_1 j_2 j_3}
    g^{i_4 j_4} \epsilonup_{j_1 j_2 j_3 j_4} \ldots \\
    &=
    \epsilonup^{i_1}
    \epsilonup_{j_1}{}^{i_2}
    \epsilonup_{j_1 j_2}{}^{i_3}
    \epsilonup_{j_1 j_2 j_3}{}^{i_4}
\end{align*}

The scope should probably be reduced to the metric tensor like so and the
Levi-Civita symbol has indices running up to $n$:
\[
    \epsilonup_{i_1 \ldots i_n}
    := \sbr{\prod_{k=1}^n g^{i_k j_k}} \epsilonup_{j_1 \ldots j_n}.
\]

The second hint looks strange as well. The free indices are $i$ and $i_2$ to
$i_n$ on the first Levi-Civita symbol and $j$ and $j_2$ to $j_n$ on the second
one. The right hand side of the equation only has $i$ and $j$ as free indices.
The second symbol should probably have $i_2$ to $i_n$ as well, right?

We now start with the actual problem. We are to compute
\begin{align*}
    \deltaup \phi
    &= \sgn(\tens g) [-1]^{n[p+1]+1} \hodge \dif \hodge \phi.
    \intertext{%
        We may assume that $\phi$ is a 1-form and therefore $p = 1$ here.
        The application of the right Hodge dual is first. In order to use
        Equation~(1) from the problem set we write $\phi$ in components.
    }
    &= \sgn(\tens g) [-1]^{2n+1} \hodge \dif \del{\phi_i \hodge \dif x^i}
    \intertext{%
        For any $n$, the power of $-1$ will always be odd. We can therefore
        just write this as a single minus sign. Now we can apply the definition
        of the Hodge dual.
    }
    &= - \sgn(\tens g) \hodge \dif \frac{\sqrt{|\tens
    g|} \phi_i}{[n-1]!} \epsilonup^{i}{}_{j_2 \ldots j_n}
    \bigwedge_{k=2}^n \dif x^{j_k}
    \intertext{%
        We write the exterior derivative in terms of components.
    }
    &= - \sgn(\tens g) \hodge \frac{\partial_\mu \sqrt{|\tens
    g|} \phi_i}{[n-1]!} \epsilonup^{i}{}_{j_2 \ldots j_n}
    \dif x^\mu \wedge \bigwedge_{k=2}^n \dif x^{j_k}
    \intertext{%
        Then we apply the second Hodge dual which brings us back to a low-order
        form.
    }
    &= - \sgn(\tens g) \frac{\sqrt{|\tens g|}}{2!}
    \partial_\mu \sqrt{|\tens g|} \phi_i
    \epsilonup^{i}{}_{j_2 \ldots j_n}
    \epsilonup^{\mu j_2 \ldots j_n}
    \intertext{%
        The two Levi-Civita symbols can be contracted using the identity given
        on the problem set.
    }
    &= - \sgn(\tens g) \frac{\sqrt{|\tens g|}}{2!}
    \frac{\partial_\mu \sqrt{|\tens g|} \phi_i}{[n-1]!} g^{i\mu} \\
    &= - \sgn(\tens g) \frac{\sqrt{|\tens g|}}{2!}
    \partial_i \sqrt{|\tens g|} \phi_i
\end{align*}
That in the neighborhood, but not the correct result.

\begin{question}
    The coderivative poses a notational problem: We have the variation which was
    denoted with $\deltaup$. Now we have the coderivative which goes by the same
    symbol. In general relativity, the variation is needed to derive the equations
    of motions (geodesic equation, Maxwell's field equations, Einstein's field
    equations). In a geometric treatment of general relativity the coderivative
    probably plays a role as well. We only have so many variations of the lowercase
    “delta”: $\delta \deltaup \vec\delta \tens\delta$. How can one make it clear
    which one to use?
\end{question}

\section{Vector calculus in $\R^3$}
\label{homework:3}

We will use $\flat$ (“flat”) and $\sharp$ (“sharp”) instead of $J$ and $J\inv$
respectively since those look cooler. Except when used as superscript and
printed with an ink jet printer like this hardcopy. Sorry. You can get the PDF
file online\footnote{\url{
http://martin-ueding.de/de/studies/msc_physics/geometry/index.html}}, if you
like.

The metric tensor for spherical polar coordinates is
\[
    \tens g =
    \begin{pmatrix}
        1 & 0 & 0 \\
        0 & r^2 & 0 \\
        0 & 0 & r^2 \sin(\theta)^2
    \end{pmatrix}.
\]
The square root of the determinant is the well known Jacobian of spherical
polar coordinates, $r^2 \sin(\theta)$.

\paragraph{Divergence}

The divergence is defined as
\begin{align*}
    \divergence \vec v
    &:= - \deltaup v^\flat.
    \intertext{%
        Now we have a 1-form $\phi = v^\flat$ and we already know what the
        coderivative of this thing is from Problem~\ref{homework:2}:
        “\nameref{homework:2}”.
    }
    &= \frac{1}{\sqrt{|g|}} \pd{}{x_i} \sqrt{|g|} v^\flat_i
    \intertext{%
        We can insert the Jacobian.
    }
    &= \frac{1}{r^2 \sin(\theta)} \pd{}{x_i} r^2 \sin(\theta) v^\flat_i
    \intertext{%
        Now we have to perform the summation over the coordinates.
    }
    &= \frac{1}{r^2 \sin(\theta)} \sbr{
        \pd{}{r} r^2 \sin(\theta) v^\flat_1
        + \pd{}{\theta} r^2 \sin(\theta) v^\flat_2
        + \pd{}{\phi} r^2 \sin(\theta) v^\flat_3
    }
    \intertext{%
        We can cancel some parts since not everything is involved in the
        product rules.
    }
    &= \frac{1}{r^2} \pd{}{r} r^2 v^\flat_1
    + \frac{1}{\sin(\theta)} \pd{}{\theta} \sin(\theta) v^\flat_2
    + \pd{}{\phi} v^\flat_3
    \intertext{%
        Now we apply the product rule to the terms. It does not make it more
        clear, though.
    }
    &= \frac{2}{r} v^\flat_1 + v^\flat_{1,1}
    + \cot(\theta) v^\flat_2 + v^\flat_{2,2}
    + v^\flat_{3,3}
\end{align*}
That is wrong.

\paragraph{Curl}

\begin{align*}
    \curl \vec v
    &:= \sbr{\hodge \dif \vec v^\flat}^\sharp
    \intertext{%
        We expand the form into its components.
    }
    &= \sbr{\hodge \dif v^\flat_i \dif x^i}^\sharp
    \intertext{%
        We apply the exterior derivative. Only derivatives of $v$ contribute.
    }
    &= \sbr{\hodge v^\flat_{i,j} \dif x^j \wedge \dif x^i}^\sharp
    \intertext{%
        Next we apply the Hodge dual onto the 2-form. This will yield a 1-form.
    }
    &= \sbr{\frac{\sqrt{|g|}}{1!} \epsilonup^{ji}{}_{k} v^\flat_{i,j} \dif x^k}^\sharp
    \intertext{%
        All operators are done, we can now rewrite this in the basis of
        vectors.
    }
    &= \sqrt{|g|} \epsilonup^{jik} v^\flat_{i,j} \partial_k
    \intertext{%
        If we look at the previous expression with Cartesian coordinates in
        mind, one can indeed see the curl of a vector field. If we now apply
        spherical polar coordinates, we have the following.
    }
    &= r^2 \sin(\theta) \epsilonup^{jik} v^\flat_{i,j} \partial_k
\end{align*}
That is also wrong.

\section{Laplace}
\label{homework:4}

\subsection{Laplacian}

\begin{align*}
    \laplace \alpha
    &= - [\dif + \deltaup]^2 \alpha \\
    &= - [\dif^2 + \dif \, \deltaup + \deltaup\dif + \deltaup^2] \alpha
    \intertext{%
        The outer terms will not contribute anything since $\dif^2 = 0$ and
        $\deltaup^2$ will generate an $n+1$ form at some point. Only the mixed
        terms contribute.
    }
    &= - [\dif \, \deltaup + \deltaup\dif] \alpha \\
    \intertext{%
        We apply the left operator of each summand to $\alpha$.
    }
    &= + \dif \alpha_i{}^{,i}
    + \deltaup \alpha_{i,j} \dif x^j \wedge \dif x^i
    \intertext{%
        Then we apply the second operator.
    }
    &= + \alpha_i{}^{,i}{}_{,j} \dif x^j
    - \hodge \dif \hodge \alpha_{i,j} \dif x^j \wedge \dif x^i \\
    &= + \alpha_i{}^{,i}{}_{,j} \dif x^j
    - \hodge \dif \frac{1}{[n-2]!} \alpha_{i,j}
    \epsilonup^{ji}{}_{l_3 \ldots l_n} \bigwedge_{m=3}^n \dif x^{l_m}
    \intertext{%
        We apply the exterior derivative.
    }
    &= + \alpha_i{}^{,i}{}_{,j} \dif x^j
    - \hodge \frac{1}{[n-2]!} \alpha_{i,j,k}
    \epsilonup^{ji}{}_{l_3 \ldots l_n} \dif x^k \wedge \bigwedge_{m=3}^n \dif x^{l_m}
    \intertext{%
        And then we perform the second Hodge dual.
    }
    &= + \alpha_i{}^{,i}{}_{,j} \dif x^j
    - \frac{1}{[n-2]!} \alpha_{i,j,k}
    \epsilonup^{ji}{}_{l_3 \ldots l_n}
    \epsilonup^{k l_3 \ldots l_n}_{\mu} \dif x^\mu
    \intertext{%
        We contract most of the indices of the Levi-Civita symbols.
    }
    &= + \alpha_i{}^{,i}{}_{,j} \dif x^j
    - \alpha_{i,j,k}
    \sbr{\deltaup^j_k \deltaup^i_\mu - \deltaup^j_\mu \deltaup^i_k} \dif x^\mu
    \intertext{%
        We expand the bracket into two terms. We need to add a minus sign to
        the second term, otherwise it would not work out. We have no idea where
        we missed that one.
    }
    &= + \alpha_i{}^{,i}{}_{,j} \dif x^j
    + \alpha_{i,j}{}^{,j} \dif x^i - \alpha_{i,j}{}^{,i} \dif x^j
    \intertext{%
        Only one term remains after cheating.
    }
    &= \alpha_{i,j}{}^{,j} \dif x^i
\end{align*}

\subsection{Symmetric}

\emph{Missing}

\end{document}

% vim: spell spelllang=en tw=79
