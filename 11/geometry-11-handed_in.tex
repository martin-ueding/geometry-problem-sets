% Copyright © 2015 Martin Ueding <dev@martin-ueding.de>

\documentclass[11pt, english, fleqn, DIV=15, headinclude, BCOR=1cm]{scrartcl}

\usepackage[bibatend]{../header}

\usepackage{tikz}

\usepackage[tikz]{mdframed}
\newmdtheoremenv[%
    backgroundcolor=black!5,
    innertopmargin=\topskip,
    splittopskip=\topskip,
]{theorem}{Theorem}[section]

\usepackage{../my-boxes}

\usepackage{enumerate}

\hypersetup{
    pdftitle=
}

\newcounter{totalpoints}
\newcommand\punkte[1]{#1\addtocounter{totalpoints}{#1}}

\newcounter{problemset}
\setcounter{problemset}{11}

\subject{Geometry in Physics}
\ihead{Geometry in Physics -- Problem Set \arabic{problemset}}

\title{Problem Set \arabic{problemset}}

\publishers{Group 1 -- Jens Boos}
\ofoot{Group 1 -- Jens Boos}

\author{
    Martin Ueding \\ \small{\href{mailto:mu@martin-ueding.de}{mu@martin-ueding.de}}
    \and
    Paul Manz \\ \small{\href{mailto:paul.manz@dreiacht.de}{paul.manz@dreiacht.de}}
}
\ifoot{Martin Ueding, Paul Manz}

\ohead{\rightmark}

\usepackage{multicol}

\renewcommand\thesubsection{\thesection.\alph{subsection}}

\begin{document}

\maketitle

\vspace{3ex}

\begin{center}
    \begin{tabular}{rrr}
        problem & achieved points & possible points \\
        \midrule
        \nameref{homework:1} & & \punkte{30} \\
        \nameref{homework:2} & & \punkte{20} \\
        \midrule
        Total & & \arabic{totalpoints}
    \end{tabular}
\end{center}

\section{Lie algebra of matrix groups}
\label{homework:1}

Are we talking about the abstract algebra of the groups or rather a matrix
representation in a given dimension? The groups we look at are matrix groups of
a given dimension. They have a “natural” matrix representation, which is the
one used before generalizing those matrices to a group. The algebra can be
written down in terms of those matrices as well. We will start with the matrix
representation in $n$ dimensions of the groups as well.

The basis of the Lie algebra are the generators, and we will take the
physicist's convention of hermitian generators for compact groups here. We
assume that the elements of the group are always exponentiated from some
algebra element, i.e. that they are connected to the identity element $e$. Then
we can just take the first order derivative at the identity, which means that
we look at $T_e G$ which will give us $\mathfrak g$.

\subsection{General linear group}

The general linear group contains literally all matrices with $n$ dimensions
over the given field, which is $\R$ here. We can parameterize any matrix by its
components. The matrix is given by:
\[
    \mat A\del{\{ \alpha^{i}{}_{j} \}_{ij}} = \{ \alpha^{i}{}_{j} \}_{ij}.
\]
The matrix is just given by the matrix of the parameters.

We can obtain the generators:
\[
    \mat T_{i}{}^{j}
    = - \iup \pd{\mat A\del{\{ \alpha^{i}{}_{j} \}_{ij}}}{\alpha^{i}{}_{j}}(\tens 0)
    = - \iup \ev_i \otimes \ev^j,
\]
where we have used the explicit form of the matrices in the last step, of
course. Those generators are not hermitian. The group in question is not
compact, so this is normal.

The matrices only depend linearly on the parameters, so the generators already
form the basis for the whole group as well. The algebra can then be written as
\[
    \mathfrak{gl}(n, \R) =
    \cbr{
        - \iup \alpha^{i}{}_{j} \ev_i \otimes \ev^j
        \colon
        \alpha^{i}{}_{j} \in \R
    }.
\]

From here, we can reach all the elements of the group with the exponential map:
\[
    g(\mat\alpha) = \exp\del{\iup \alpha^i{}_j \mat T_i{}^j}.
\]

\subsection{Special linear group}

The special linear group has the constraint that the determinant has to be one.
On the previous exercise sheet we derived the useful relation $\det\exp(\mat A)
= \exp(\tr \mat A)$. In our case we can take $\mat A \in \mathfrak{sl}$ and see
that $\exp(\mat A) \in \mathrm{GL}$, then. A unit determinant then corresponds
to generators which do not have a trace.

We remove the trace from the generators that we had before by just changing the
parametrization a bit. All the generators which have an entry on the diagonal
will get an additional $-1$ in the very first element. They now look like this:
\[
    \mat T_{i}{}^{j}
    = - \iup \sbr{\ev_i \otimes \ev^j - \deltaup_i^j \ev_1 \otimes \ev^1}.
\]
This makes the generator $\mat T_1{}^1$ useless. The unit determinant gave us
one constraint, so the group now has one dimension less and therefore also
needs one generator less.

\subsection{Special orthogonal group}

Special orthogonal matrices are such that $\mat O\inv = \mat O^\mathrm T$. Here
the exponential map is of great help.
Let the algebra element $\mat o$ be the generator of the group element $\mat
O$. Then we can rewrite the defining property.
\begin{align*}
    \mat O\inv &= \mat O^\mathrm T \\
    \exp(\iup \mat o)\inv &= \exp(\iup \mat o)^\mathrm T \\
    \exp(-\iup \mat o) &= \exp(\iup \mat o^\mathrm T) \\
    -\mat o &= \mat o^\mathrm T
\end{align*}
We now want antisymmetric generators. The group is a compact lie group because
the parameters which we are going to introduce later are defined on compact
support. Compact groups can have hermitian generators. This is the first
compact group, so the generators actually are hermitian and purely imaginary,
therefore antisymmetric.

Rotations can be parametrized by angles around all the axes in the space. One
such active rotation around $\Theta^i$ can be written as such:
\[
    \mat O(\Theta^i) =
    \begin{pmatrix}
        1 & 0 & \cdots & 0 & \cdots & 0 & \cdots & 0 & 0 \\
        0 & 1 & \cdots & 0 & \cdots & 0 & \cdots & 0 & 0 \\
        0 & 0 & \ddots & \cos(\Theta^i) & \cdots & -\sin(\Theta^i) & \cdots & 0 & 0 \\
        0 & 0 & \cdots & 0 & \cdots & 0 & \cdots & 0 & 0 \\
        0 & 0 & \cdots & \sin(\Theta^i) & \cdots & \cos(\Theta^i) & \ddots & 0 & 0 \\
        0 & 0 & \cdots & 0 & \cdots & 0 & \cdots & 1 & 0 \\
        0 & 0 & \cdots & 0 & \cdots & 0 & \cdots & 0 & 1 \\
    \end{pmatrix},
\]
there are ones on the diagonal except for the elements where the actual
rotation takes place. The generators of this group are then simply
\[
    \mat T_{ij} = -\iup \ev_{[i} \otimes \ev_{j]}.
\]
In three dimensions, there are only three distinct rotations. In four
dimensions there are already six possible rotations, which we know from the
Lorentz group with its six (pseudo-)rotations. The rotation angles should be
given as an antisymmetric matrix instead of a vector. We regularly use a vector
in three dimensions, which is again a hidden Hodge dual nobody told us about
before.

The components of the generators can be written as
\[
    \sbr{T_{ij}}^{ab} = - \iup \deltaup^a_{[i} \deltaup^b_{j]}
\]
or alternatively
\[
    \sbr{T_{ij}}_{ab} = - \iup \sbr{
        \eta_{ai} \eta_{bj} - \eta_{aj} \eta_{bi}
    }
\]
which can be generalized to any matrix group $\mathrm{SO}(n-p, p, \R)$ if the
correct signature of the metric tensor is inserted. The first version is also
given similarly by \textcite[(3.18)]{Peskin/QFT/1995}.

\subsection{Unitary group}

Unitary matrices $\mat U$ have the property $\mat U^\dagger = \mat U\inv$. This
can be used with the exponential map like with the orthogonal matrices to give
a constraint for the generator $\mat u$.
\begin{align*}
    \mat U^\dagger &= \mat U\inv \\
    \exp(\iup \mat u)^\dagger &= \exp(\iup \mat u)\inv \\
    \exp(- \iup \mat u^\dagger) &= \exp(- \iup \mat u) \\
    \mat u^\dagger &= \mat u
\end{align*}
So again we have have hermitian generators for this group. The group itself is
not compact though, as can be seen later when parametrizing it.

We know that the Pauli matrices are one basis for the \emph{special} unitary
algebra in the two dimensional representation and the Gell-Mann matrices the
basis in the three dimensional representation. Since we regard the general
unitary group, we need one additional real parameters such that the determinant
can have any argument, while the modulus is still 1. Adding the identity matrix
(or in physicist's convention the matrix $- \iup \mat 1_n$) to the generators
will basically give us $\SU(n) \times \mathrm U(1) = \mathrm U(n)$.

The real part of the trace of the generator does not have to be zero as we are
allowing a complex phase for this group. This means that our generators fall in
three categories.

\begin{description}
    \item[Imaginary off-diagonal]
        There are generators which looks like these:
        \[
            \begin{pmatrix}
                0 & - \iup \alpha & 0 \\
                \iup \alpha & 0 & 0 \\
                0 & 0 & 0
            \end{pmatrix}
            \begin{pmatrix}
                0 & 0 & - \iup \alpha \\
                0 & 0 & 0 \\
                \iup \alpha & 0 & 0
            \end{pmatrix}
            \begin{pmatrix}
                0 & 0 & 0 \\
                0 & 0 & - \iup \alpha \\
                0 & \iup \alpha & 0 \\
            \end{pmatrix}
            .
        \]
        There are $n[n-1]/2$ of those. This can be seen since a $n \times n$
        matrix has $n^2$ elements. We take away the diagonal and are at
        $n[n-1]$. Then only the upper or lower elements are unique, the other
        half is determined since the matrix is supposed to be hermitian.

    \item[Real on-diagonal]
        Since we do not have to take care of the trace, we can add $n$
        generators of the form
        \[
            \begin{pmatrix}
                1 & 0 & 0 \\
                0 & 0 & 0 \\
                0 & 0 & 0
            \end{pmatrix}
            \eqnsep
            \begin{pmatrix}
                0 & 0 & 0 \\
                0 & 1 & 0 \\
                0 & 0 & 0
            \end{pmatrix}
            \eqnsep
            \begin{pmatrix}
                0 & 0 & 0 \\
                0 & 0 & 0 \\
                0 & 0 & 1
            \end{pmatrix}
            .
        \]

    \item[Real off-diagonal]
        And there are another $n[n-1]/2$ generators for the real off-diagonal
        entries.
        \[
            \begin{pmatrix}
                0 & 1 & 0 \\
                1 & 0 & 0 \\
                0 & 0 & 0
            \end{pmatrix}
            \eqnsep
            \begin{pmatrix}
                0 & 0 & 1 \\
                0 & 0 & 0 \\
                1 & 0 & 0
            \end{pmatrix}
            \eqnsep
            \begin{pmatrix}
                0 & 0 & 0 \\
                0 & 0 & 1 \\
                0 & 1 & 0
            \end{pmatrix}
        \]
\end{description}

Together we have $n^2$ generators with real parameters. This means that the
dimension of the algebra $\mathfrak u(n)$ has half the dimension compared to
$\gl(n, \R)$. The group $\SU(n)$ has $n^2 - 1$ generators, which fits with the
three Pauli and eight Gell-Mann matrices. As always, the algebra is the space
spanned by those $n^2$ generators.

\subsection{Lorentz group}

With the introduction of the metric tensor for the special orthogonal group, we
already derived this for the Lorentz group as well. Just set the metric tensor
to $\mat\eta = \diag(1, -1, -1, -1)$.

Using Greek indices we have the six generators
\[
    \sbr{T^{\mu\nu}}^{\rho\lambda} = - \iup \sbr{
        \eta^{\rho\mu} \eta^{\lambda\nu} - \eta^{\rho\nu} \eta^{\lambda\mu}
    }.
\]
The corresponding angles go into an angle two-form $\omega$. The elements
$\omega_{0i}$ correspond to pseudo-rotations or boosts, whereas the elements
$\omega_{ij}$ correspond to rotations about the axes $k$ (with $\epsilon_{ijk}$
in mind).

\subsection{Heisenberg group}

Thankfully the Heisenberg group is already given with parameters. So we can
just write down the generators:
\[
    \mat T_1 = - \iup \ev_1 \otimes \ev^2
    \eqnsep
    \mat T_3 = - \iup \ev_1 \otimes \ev^3
    \eqnsep
    \mat T_3 = - \iup \ev_2 \otimes \ev^3
\]
And then the algebra simply is
\[
    \mathfrak h =
    \cbr{
        - \iup \alpha^{i} \mat T_i
        \colon
        \alpha^i \in \R, i \in \{ 1, 2, 3 \}
    }
\]


\section{Field matter coupling in quantum mechanics}
\label{homework:2}

\subsection{Einstein-Lorentz force}

The Lagrangian with all explicit functional dependence written down is
\[
    L(\vec x, \dot{\vec x}) = \frac12 \eta_{ij} \dot x^i \dot x^j
    - \phi(\vec x, t) + \dot x^i A_i(\vec x, t).
\]
We use $\tens \eta$ instead of $\tens g$ as long as we are in coordinate frame
without curvature.

The Euler-Lagrange equation for this is $\deltaup L = 0$ where $\deltaup$ means
the variation. We have
\begin{align*}
    0 &= \pd{L}{x^i} - \od{}t \pd{L}{\dot x^i} \\
    \intertext{%
        We compute the partial derivatives.
    }
    0 &= - \phi_{,i} + \dot x^j A_{j,i} - \od{}t \sbr{\dot x_i + A_i(\vec x, t)} \\
    \intertext{%
        The total time derivative will lead to a chain rule in $\dot x$.
    }
    0 &= - \phi_{,i} + \dot x^j A_{j,i} - \ddot x_i - A_{i,j} \dot x^j - \dot
    A_i \\
    \intertext{%
        We move $\ddot x_i$ to the other side.
    }
    \ddot x_i &= - \phi_{,i} + \dot x^j A_{j,i} - A_{i,j} \dot x^j - \dot
    A_i \\
    \intertext{%
        With a bit of vector calculus magic we can regroup those terms. It is
        easiest to start with the result and check using the Levi-Civita
        symbols that it matches the expression in the line above.
    }
    \ddot x_i &= \underbrace{- \vnabla \phi}_{\vec E} + \dot{\vec x} \times
    \underbrace{[\vnabla \times \vec A]}_{\vec B} \\
    \ddot x_i &= \vec E + \dot{\vec x} \times \vec B
\end{align*}
And that is the Einstein-Lorentz equation.

\subsection{Hamiltonian}

The canonical momenta are
\[
    \pi_i = \pd{L}{\dot x^i} = \dot x_i + A_i(\vec x, t).
\]
We assemble the Hamiltonian using the Legendre transformation.
\begin{align*}
    H &= \pi_i \dot x^i - L \\
    &= \pi_i \dot x^i - \frac12 \dot x_i \dot x^i - \phi(\vec x, t) + \dot
    x^i A_i(\vec x, t) \\
    &= \pi_i \sbr{\pi^i - A^i(\vec x, t)}
    - \frac12 \sbr{\pi_i - A_i(\vec x, t)} \sbr{\pi^i - A^i(\vec x, t)}
    + \phi(\vec x, t)
    - \sbr{\pi^i - A^i(\vec x, t)} A_i(\vec x, t) \\
    \intertext{%
        We can factor the first and last term together and have two times the
        second term. Together we only have the positive second term left.
    }
    &= \frac12 \sbr{\pi_i - A_i} \sbr{\pi^i - A^i} + \phi
\end{align*}

\subsection{Gauge transformation}

The time dependant Schrödinger equation is
\[
    \iup \pd{}t \psi(\vec x, t) = H(\vec x, t) \psi(\vec x, t).
\]
Inserting our expression from the previous part and replacing $\pi_i$ with $-
\iup \pd{}t$, we arrive at the given Schrödinger equation.

We start with the transformation of the magnetic potential.
\begin{align*}
    \iup \dot\psi
    &= \sbr{\frac12 \sbr{- \iup \partial_i - A_i' + \theta_{,i}} \sbr{- \iup
    \partial^i - A^{i\prime} + \theta^{,i}} + \phi' + \dot\theta} \psi(\vec x, t) \\
    \intertext{%
        The terms with $\theta$ get factored out. 
    }
    &= \sbr{
        \frac12
        \sbr{
            \sbr{- \iup \partial_i - A_i'}
            \sbr{- \iup \partial^i - A^{i\prime}}
            +
            \sbr{- \iup \partial_i - A_i'}
            \theta^{,i}
            +
            \theta_{,i}
            \sbr{- \iup \partial^i - A^{i\prime}}
            +
            \theta_{,i}
            \theta^{,i}
        }
        + \phi' + \dot\theta
    } \psi
    \intertext{%
        We extract the original Hamiltonian.
    }
    &= \sbr{
        H' +
        \frac12
        \sbr{
            \sbr{- \iup \partial_i - A_i'}
            \theta^{,i}
            +
            \theta_{,i}
            \sbr{- \iup \partial^i - A^{i\prime}}
            +
            \theta_{,i}
            \theta^{,i}
        }
        + \dot\theta
    } \psi \\
    \intertext{%
        The remaining inner brackets are expanded, keeping the product rule in
        mind.
    }
    &= \sbr{
        H' +
        \frac12
        \sbr{
            - \iup \theta^{,i}{}_{,i} - A_i' \theta^{,i}
            - \iup \theta^{,i} \partial_i
            - \iup \theta_{,i} \partial^i
            - \theta_{,i} A^{i\prime}
            + \theta_{,i} \theta^{,i}
        }
        + \dot\theta
    } \psi \\
    \intertext{%
        Two terms occur twice, we group them.
    }
    &= \sbr{
        H' +
        \frac12
        \sbr{
            - \iup \theta^{,i}{}_{,i}
            - 2 A_i' \theta^{,i}
            - 2 \iup \theta^{,i} \partial_i
            + \theta_{,i} \theta^{,i}
        }
        + \dot\theta
    } \psi
\end{align*}
There are terms left that we cannot get rid of. They prevent a form-invariance
of the Schrödinger equation.

The solution is to add the phase transformation to the wave function. So we
replace $\psi$ by $\exp(-\iup \theta) \psi'$. First we compute the commutator
of $H'$ with $\exp(-\iup \theta)$.
\begin{gather*}
    \sbr{H', \exp(-\iup \theta)} \psi
    = \sbr{\sbr{\frac12 \sbr{- \iup \partial_i - A_i'} \sbr{- \iup
    \partial^i - A^{\prime i}} + \phi'}, \exp(-\iup \theta)} \psi
    \intertext{%
        Every summand which does not contain a differential operator will not
        contribute to the commutator. We can therefore already eliminate
        $\phi'$.
    }
    \quad= \frac12 \sbr{\sbr{- \iup \partial_i - A_i'} \sbr{- \iup
    \partial^i - A^{\prime i}}, \exp(-\iup \theta)} \psi
    \intertext{%
        Then we expand the brackets.
    }
    \quad= \frac12 \sbr{
        - \partial_i \partial^i
        - \iup \partial_i A^{\prime i}
        - \iup A_i' \partial^i
        + A_i'A^{i\prime}
    , \exp(-\iup \theta)} \psi
    \intertext{%
        We apply the product rule in the second term.
    }
    \quad= \frac12 \sbr{
        - \partial_i \partial^i
        - \iup A^{\prime i}{}_{,i}
        - \iup A^{i\prime} \partial_i
        - \iup A_i' \partial^i
        + A_i'A^{i\prime}
    , \exp(-\iup \theta)} \psi
    \intertext{%
        Again, everything without differential operators does not contribute.
    }
    \quad= \frac12 \sbr{
        - \partial_i \partial^i
        - 2 \iup A_i' \partial^i
    , \exp(-\iup \theta)} \psi \\
    \intertext{%
        We write out the commutator in full.
    }
    \quad= - \frac12 \sbr{
        \sbr{
            \partial_i \partial^i
            + 2 \iup A_i' \partial^i
        }
        \exp(-\iup \theta) \psi
        -
        \exp(-\iup \theta)
        \sbr{
            \partial_i \partial^i
            + 2 \iup A_i' \partial^i
        }
        \psi
    } \\
    \intertext{%
        We apply the product rule for the first partial derivative.
    }
    \quad= - \frac12 \left[
        \sbr{
            - \iup \partial_i \exp(-\iup \theta) \theta^{,i} \psi
            + \partial_i \exp(-\iup \theta) \partial^i \psi
            - \iup 2 A_i' \exp(-\iup \theta) \theta^{,i} \psi
            + 2 \iup A_i' \exp(-\iup \theta) \partial^i \psi
        }
        \right.\\\qquad\left.
        -
        \exp(-\iup \theta)
        \sbr{
            \partial_i \partial^i
            + 2 \iup A_i' \partial^i
        }
        \psi
    \right]
    \intertext{%
        One term can already be cancelled.
    }
    \quad= - \frac12 \left[
        \sbr{
            - \iup \partial_i \exp(-\iup \theta) \theta^{,i} \psi
            + \partial_i \exp(-\iup \theta) \partial^i \psi
            - \iup 2 A_i' \exp(-\iup \theta) \theta^{,i} \psi
        }
        - \exp(-\iup \theta) \partial_i \partial^i \psi
    \right] \\
    \quad= - \frac12 \exp(-\iup \theta) \sbr{
        - \theta_{,i} \theta^{,i}
        - \iup \theta^{,i}{}_{,i}
        - \iup \theta^{,i} \partial_i
        + \iup \theta_{,i} \partial^i
        + \partial_i \partial^i
        - 2 \iup A_i' \theta^{,i}
        - \partial_i \partial^i
    } \psi
    \intertext{%
        We can cancel several more terms here.
    }
    \quad= - \frac12 \exp(-\iup \theta) \sbr{
        - \theta_{,i} \theta^{,i}
        - \iup \theta^{,i}{}_{,i}
        - 2 \iup A_i' \theta^{,i}
    } \psi
\end{gather*}

These terms almost match up with the extra terms we have in the Schrödinger
equation. The time derivative on the other side will give us the $- \dot
\theta$ term that we also need to cancel it in the Schrödinger equation.

\subsection{Comparison}

From what we have heard before this lecture, we would identify:

\begin{itemize}
    \item Bundle of vector spaces: Phase space
    \item Gauge group: $\mathrm U(1)$
    \item Lie algebra: $\mathfrak u(1) \simeq \R$
    \item Matter field: Particle $(\vec x, t)$
    \item Connection: Magnetic potential $\vec A$
\end{itemize}

\end{document}

% vim: spell spelllang=en tw=79
