% Copyright © 2015 Martin Ueding <dev@martin-ueding.de>

\documentclass[11pt, english, fleqn, DIV=15, headinclude, BCOR=1cm]{scrartcl}

\usepackage[bibatend]{../header}

\usepackage{tikz}

\usepackage[tikz]{mdframed}
\newmdtheoremenv[%
    backgroundcolor=black!5,
    innertopmargin=\topskip,
    splittopskip=\topskip,
]{theorem}{Theorem}[section]

\usepackage{../my-boxes}

\hypersetup{
    pdftitle=
}

\newcounter{totalpoints}
\newcommand\punkte[1]{#1\addtocounter{totalpoints}{#1}}

\newcounter{problemset}
\setcounter{problemset}{8}

\subject{Geometry in Physics}
\ihead{Geometry in Physics -- Problem Set \arabic{problemset}}

\title{Problem Set \arabic{problemset}}

\publishers{Group 1 -- Jens Boos}
\ofoot{Group 1 -- Jens Boos}

\author{
    Martin Ueding \\ \small{\href{mailto:mu@martin-ueding.de}{mu@martin-ueding.de}}
    \and
    Paul Manz \\ \small{\href{mailto:paul.manz@dreiacht.de}{paul.manz@dreiacht.de}}
}
\ifoot{Martin Ueding, Paul Manz}

\ohead{\rightmark}

\usepackage{multicol}

\renewcommand\thesubsection{\thesection.\alph{subsection}}

\begin{document}

\maketitle

\vspace{3ex}

\begin{center}
    \begin{tabular}{rrr}
        problem & achieved points & possible points \\
        \midrule
        \nameref{homework:1} & & \punkte{20} \\
        \nameref{homework:2} & & \punkte{10} \\
        \nameref{homework:3} & & \punkte{5} \\
        \nameref{homework:4} & & \punkte{10} \\
        \midrule
        Total & & \arabic{totalpoints}
    \end{tabular}
\end{center}

\section{Poynting's theorem}
\label{homework:1}

\begin{enumerate}
    \item
The field $D$ is a two-form but only has one index. In this case it is the
Hodge dual of the real thing and should have an upper index.
\begin{align*}
    W_\mathrm e
    &= \frac12 E \wedge D
    = \frac 12 E_l \dif x^l \wedge \epsilonup_{ijk} D^k \dif x^i \wedge \dif
    x^j
    = \frac 12 E_l D^k \epsilonup_{ijk} \dif x^l \wedge \dif x^i \wedge \dif
    x^j \\
    &= \frac 12 E_l D^k \epsilonup_{ijk} \epsilon^{lij} \dif x \wedge \dif y \wedge \dif
    z
    = E_l D^l \dif x \wedge \dif y \wedge \dif z
\end{align*}
And similarly we have $W_\mathrm m = H_l B^l \dif V$.

\item

Ampère's law is $\curl \vec H - \dot {\vec D} = \vec J_\mathrm f$. In
components this can be written as $\epsilonup^{ijk} H_{k,j} - \dot D^i = J^i$.
Faraday's law is $\curl \vec E + \dot{\vec B} = \vec 0$, which is
$\epsilonup^{ijk} E_{k,j} + \dot B^i = 0$ in components. The given wedge
product simply is $E \wedge H = E_i H_k \dif x^i \wedge \dif x^k$.
\begin{align*}
    \dif(E \wedge H)
    &= \partial_\mu (E_i H_k) \dif x^\mu \wedge \dif x^i \wedge \dif x^k
    = \sbr{E_{i,\mu} H_k + E_i H_{k,\mu}} \dif x^\mu \wedge \dif x^i \wedge
    \dif x^k \\
    \intertext{%
        We can split off the time derivative.
    }
    &= \sbr{\dot E_i H_k + E_i \dot H_k} \dif t \wedge \dif x^i \wedge \dif x^k
    + \sbr{E_{i,j} H_k + E_i H_{k,j}} \epsilonup^{jik} \dif x \wedge \dif y
    \wedge \dif z \\
    &= \sbr{\dot E_i H_k + E_i \dot H_k} \dif t \wedge \dif x^i \wedge \dif x^k
    + \sbr{H_i \dot B^i + E_i \dot D^i} \dif x \wedge \dif y \wedge \dif z
\end{align*}

\item
    $\dif \sigma$ is the time derivative of the field energy. It is a
    three-form, which means that it takes a time-like and two space-like
    vectors. It corresponds to the Poynting vector, which can be thought of as
    a Hodge dual of a three-form in four-space. $\sigma$ itself is an
    energy-density two-form. It cannot be the energy momentum tensor since that
    is symmetric.

\item
    The absence of charges means that all of Maxwell's equations are
    homogeneous. Then, in the sense of vectors, $\vec E \propto \vec D$ and
    $\vec H \propto \vec B$ which means that the wedge products are just the
    scalar product of the components. Depending on the choice of unit system
    they are even equal to each other, respectively.

\item
    We start with Equation~(4): $\dif \sigma = - \frac12 \partial_0 [E \wedge D
    + H \wedge B]$. As mentioned before, in the absence of charge means that we
    only have simple scalar products: $\dif \sigma = - \frac12 \partial_0 [\vec
    E^2 + \vec B^2] \dif V$. Now one integrates over the volume $V$ and applies
    Stokes's theorem to the left hand side. That's it.

\end{enumerate}

\section{Transformation properties of the metric}
\label{homework:2}

\begin{enumerate}
    \item 
        A (mathematician's\footnote{At times it becomes a bit mixed up since
        mathematicians call this bilinear form “metric”, whereas physicists
        often call the metric tensor $\tens G$ which goes into the scalar
    product “metric”: $\bracket{u, v} := \vec u^\mathrm T \tens G \vec v$. “Metric
tensor” would be a better name for this $\tens G$.}) metric, which is a bilinear mapping of two
vectors, has to be symmetric by definition. The matrix $\mat g$ which is
defined by $g_{ij} := g(\ev_i, \ev_j)$ is therefore symmetric in its indices.

    \item
        The metric is defined to be non-degenerate. This prohibits an
        eigenvalue of 0, which in turn means that $\det(\mat g - 0 \cdot \mat
        1_2)$ cannot be zero.

    \item
        A change of basis has to leave the norm invariant. We transform the
        basis vectors as given: $\ev_i' = \ev_j [\mat A\inv]^j{}_i$. Then the
        components of vectors have to transform the other way around, such that
        the vectors themselves are not altered. This means $v^{i\prime} = v^j
        A^i{}_j$. The vector is unaltered. The transformed vector is written in
        components: $\vec v' = v^{i\prime} \ev_i'$. Expanded, this is $v^j
        \ev_k A^i{}_j [\mat A\inv]^k{}_i$. The inverse of the matrix time the
        matrix is the identity, which in components is just the Kronecker
        symbol. We therefore get $v^j \ev_j$, which is the same vector as
        before. Extending this argument to tensors which are two times
        covariant, we need to apply the inverse transformation twice in order
        to preserve the action on two vectors. Writing this without indices
        gives the expression Equation~(6) on the problem set, although one
        should not use the indices $(ij)$ twice.
\end{enumerate}

\section{Lorentz group 1}
\label{homework:3}

We have $R$ from the three dimensional “natural” representation $\Gamma_3$ of
$\SO(3)$. Those transformations leave the metric $\tens 1_3$ invariant. The
group $\SO(3)$ also has the trivial $\Gamma_1$ representation where $\forall g
\in \SO(3) \colon D^{\Gamma_1}(g) = 1$. We can now build a reducible
representation using this: $D^\text R := D^{\Gamma_1} \oplus D^{\Gamma_3}.$ In
this representation, we get this $\tilde R$. This still is an element of
$\SO(3)$, albeit in a four dimensional representation.

The proper orthochronous Lorentz group contains all the elements which preserve
the orientation of time and the orientation of space. The orientation of space
is conserved by $R$ since it is chosen from the \emph{special} orthogonal
group. The conservation of time requires two steps. It is decoupled from space
since we have a reducible representation which is built up such that time and
space remain invariant subspaces in the space on which the representation acts
on. The representation of the group element in the temporal part is given using
the trivial representation, it therefore does not change anything.

\section{Lorentz group 2}
\label{homework:4}

\begin{enumerate}
    \item 
        We have given $\mat A \in L$ and we may assume that $\mat A \ev_0$ is a
        future directed time like vector. This also means that $\eta(\ev_0,
        \mat A \ev_0) > 0$, which implies that the component of $A^0{}_0$ is
        positive. This component decides the orientation of time. Since it is
        positive, we have an orthochronous Lorentz transformation.

    \item
        We have to show that the set of orthochronous Lorentz transformations
        leaves light like vectors as such. There is no way that any Lorentz
        transformation could transform a time or space like vector into a light
        like vector. This would require a change in norm from non-zero to zero,
        which is not possible with an orthogonal transformation. By the same
        argument, it is not possible to transform vectors from space to time
        like and vice versa. A time like vector can only be transformed into
        another such vector, although time reversal is possible by a
        \emph{general} Lorentz transformation.

        The orthochronous Lorentz group has the distinguishing property that
        its elements do not change the time direction when applied to vectors
        from Minkowski space. We are interested in time like vectors to begin
        with, so the set $Z^+$ is the right choice. Since $\mat A \in L$ must
        not change the metric, the condition $\eta(\mat A \ev_0, \mat A \ev_0)
        > 0$ directly follows from $\eta(\ev_0, \ev_0) > 0$. The condition
        $\eta(\ev_0, \mat A \ev_0) = A^0{}_0 > 0$ that we impose by requiring
        that $\mat A$ maps $Z^+$ to itself ensures that we do not have a time
        reversal. Therefore this set is the orthochronous Lorentz group.
\end{enumerate}

\end{document}

% vim: spell spelllang=en tw=79
