% Copyright © 2015 Martin Ueding <dev@martin-ueding.de>

\documentclass[11pt, english, fleqn, DIV=15, headinclude, BCOR=1cm]{scrartcl}

\usepackage[bibatend]{../header}

\usepackage{tikz}

\usepackage[tikz]{mdframed}
\newmdtheoremenv[%
    backgroundcolor=black!5,
    innertopmargin=\topskip,
    splittopskip=\topskip,
]{theorem}{Theorem}[section]

\usepackage{../my-boxes}

\usepackage{enumerate}
\usepackage{lastpage}

\hypersetup{
    pdftitle=
}

\newcounter{totalpoints}
\newcommand\punkte[1]{#1\addtocounter{totalpoints}{#1}}

\newcounter{problemset}
\setcounter{problemset}{12}

\subject{Geometry in Physics}
\ihead{Geometry in Physics -- Problem Set \arabic{problemset}}

\title{Problem Set \arabic{problemset}}

\publishers{Group 1 -- Jens Boos}
\ofoot{Group 1 -- Jens Boos}

\author{
    Martin Ueding \\ \small{\href{mailto:mu@martin-ueding.de}{mu@martin-ueding.de}}
    \and
    Paul Manz \\ \small{\href{mailto:paul.manz@dreiacht.de}{paul.manz@dreiacht.de}}
}
\ifoot{Martin Ueding, Paul Manz}

\ohead{\rightmark}

\usepackage{multicol}

\renewcommand\thesubsection{\thesection.\alph{subsection}}

\begin{document}

\maketitle

\vspace{3ex}

\begin{center}
    \begin{tabular}{rrr}
        problem & achieved points & possible points \\
        \midrule
        \nameref{homework:1} & & \punkte{30} \\
        \nameref{homework:2} & & \punkte{20} \\
        \midrule
        Total & & \arabic{totalpoints}
    \end{tabular}
\end{center}

\vspace{3ex}

\begin{center}
    \begin{small}
        This document consists of \pageref{LastPage} pages.
    \end{small}
\end{center}

\section{Lie groups}
\label{homework:1}

\section{Lie braket}
\label{homework:2}

\subsection{Lie derivative}

We write this with a test function $f$ to make the product rule more clear.
\begin{align*}
    [v, w]f
    &= v(w(f)) - w(v(f)) \\
    \intertext{%
        We expand the two vector fields.
    }
    &= v^i \partial_i w^j \partial_j f - w^j \partial_j v^i \partial_i f \\
    \intertext{%
        Now we apply the product rule.
    }
    &= v^i w^j{}_{,i} \partial_j f
    + v^i w^j \partial_i \partial_j f
    - w^j v^i{}_{,j} \partial_i f
    - w^j v^i \partial_j \partial_i f \\
    \intertext{%
        The partial derivatives commute, so the second and third term cancel.
    }
    &= v^i w^j{}_{,i} \partial_j f - w^j v^i{}_{,j} \partial_i f \\
\end{align*}
Extracting $f$ will give the desired result.

\subsection{Properties}

\begin{enumerate}
    \item
        The derivates are linear operators. So the bilinearity follows from
        that.

    \item
        The skew-symmetry can be seen from Equation~(2), just add a minus sign
        to change the order of $v$ and $w$.

    \item
        The Jacoby identity can be shown by plugging in and seeing all the
        terms cancel. We are not going to do that, though.
\end{enumerate}

\subsection{Lie derivative}

We either have to apply a test function ore keep the product rule in mind. We
do it without an explicit test function.
\begin{align*}
    \mathscr L_{fv} w
    &= [fv, w] \\
    &= fv(w) - w(fv)
    \intertext{%
        We expand the vector fields.
    }
    &= f v^i \partial_i w^j \partial_j - w^j \partial_j f v^i \partial_i
    \intertext{%
        We apply the product rule.
    }
    &= f v^i w^j{}_{,i} \partial_j + f v^i w^j \partial_i \partial_j
    - w^j f_{,j} v^i \partial_i - w^j f v^i{}_{,j} \partial_i - w^j f v^i
    \partial_j \partial_i
    \intertext{%
        The second and last term cancel each other.
    }
    &= f v^i w^j{}_{,i} \partial_j
    - w^j f_{,j} v^i \partial_i - w^j f v^i{}_{,j} \partial_i
    \intertext{%
        We swap the second and third term.
    }
    &= f v^i w^j{}_{,i} \partial_j
    - w^j f v^i{}_{,j} \partial_i
    - w^j f_{,j} v^i \partial_i
    \intertext{%
        Then we can write the first two terms as a Lie bracket. The last term
        can be written without indices again.
    }
    &= f [v, w]
    - w(f) \, v
\end{align*}

The second identity follows from the skew-symmetry.
\[
    \mathcal L_v(fw)
    = [v, fw]
    = - [fw, v]
    = - f[w, v] + v(f) \, w
    = f[v, w] + v(f) \, w
\]

\subsection{Pushforward}

In this part we have the following definitions:

\begin{itemize}
    \item Differentiable manifolds $M$, $N$
    \item Point $p \in M$
    \item Point $q \in N$
    \item Smooth map $f \colon M \to N$
    \item Pushforward $f_* \colon T_p M \to T_{f(p)} M$
    \item Test function $g \colon N \to \R$ and $g \in C^\infty(N)$.
    \item Vector fields $v, w \in TM$.
\end{itemize}

The pushforward is given as
\[
    [f_* v](g) = v(g \circ f).
\]
Since $g$ takes an argument from the space $N$, we can write
\[
    [f_* v](g)(p) = v(g(f(p))) = [v \circ g \circ f](p).
\]

We start by applying the pushforward to the Lie bracket as a whole.
\begin{align*}
    \sbr{f_* [v, w]}(g)(p)
    &= [v, w](g \circ f)(p)
    \intertext{%
        Then we expand the Lie bracket, leaving the evaluation with $p$ on the
        outside.
    }
    &= \sbr{v(w(g \circ f)) - w(v(g \circ f))}(p)
    \intertext{%
        We can write it with less parentheses using the concatenation symbol.
    }
    &= [v \circ w \circ g \circ f - w \circ v \circ g \circ f](p)
    \intertext{%
        To ease the notation, we introduce a notation similar to “$+ \cc$” and
        “$+ \hc$” where we write “swapped” to mean the same terms with $v$ and
        $w$ swapped.
    }
    &= [v \circ w \circ g \circ f - \text{swapped}](p)
    \intertext{%
        We apply both terms to $p$ individually.
    }
    &= [v \circ w \circ g \circ f](p) - \text{swapped} \\
    \intertext{%
        The terms inside of the bracket are a pushforward.
    }
    &= \sbr{v \circ [f_* w](g)}(p) - \text{swapped} \\
    \intertext{%
        This is the point in the derivation where both ends are glued together
        with black magic. In order to get two pullbacks out, we need another
        $f$ somewhere. That does not makes sense, though, since $p \in M$ and
        $f(p) \in N$. We cannot apply $f$ twice. We probably have a mistake
        somewhere else such that this does not work out nicely here.
    }
    &= \sbr{v \circ [f_* w](g)}(f(p)) - \text{swapped} \\
    \intertext{%
        This is another pushforward, this time for $v$.
    }
    &= \sbr{[f_* v]\del{[f_* w](g)}}(p) - \text{swapped} \\
    \intertext{%
        Taking both terms together, we get another Lie bracket.
    }
    &= [f_* v, f_* w](g)(p)
\end{align*}

\end{document}

% vim: spell spelllang=en tw=79
