% Copyright © 2015 Martin Ueding <dev@martin-ueding.de>

\documentclass[11pt, english, fleqn, DIV=15, headinclude, BCOR=1cm]{scrartcl}

\usepackage[bibatend]{../header}

\usepackage{tikz}

\usepackage[tikz]{mdframed}
\newmdtheoremenv[%
    backgroundcolor=black!5,
    innertopmargin=\topskip,
    splittopskip=\topskip,
]{theorem}{Theorem}[section]

\usepackage{../my-boxes}

\hypersetup{
    pdftitle=
}

\newcounter{totalpoints}
\newcommand\punkte[1]{#1\addtocounter{totalpoints}{#1}}

\newcounter{problemset}
\setcounter{problemset}{6}

\subject{Geometry in Physics}
\ihead{Geometry in Physics -- Problem Set \arabic{problemset}}

\title{Problem Set \arabic{problemset}}

\publishers{Group 1 -- Jens Boos}
\ofoot{Group 1 -- Jens Boos}

\author{
    Martin Ueding \\ \small{\href{mailto:mu@martin-ueding.de}{mu@martin-ueding.de}}
    \and
    Paul Manz \\ \small{\href{mailto:paul.manz@dreiacht.de}{paul.manz@dreiacht.de}}
}
\ifoot{Martin Ueding, Paul Manz}

\ohead{\rightmark}

\usepackage{multicol}

\renewcommand\thesubsection{\thesection.\alph{subsection}}

\begin{document}

\maketitle

\vspace{3ex}

\begin{center}
    \begin{tabular}{rrr}
        problem & achieved points & possible points \\
        \midrule
        \nameref{homework:1} & & \punkte{10} \\
        \nameref{homework:2} & & \punkte{24} \\
        \nameref{homework:3} & & \punkte{16} \\
        \midrule
        Total & & \arabic{totalpoints}
    \end{tabular}
\end{center}

\section{Gauss theorem}
\label{homework:1}

\subsection{Stokes' theorem}

\begin{align*}
    \int_{\partial\sigma} \phi
    &= \int_{\partial [0,1]^3} Q^* \phi
    \intertext{%
        We replace $\phi$ with $\hodge \vec v$:
        \[
            v^k := \frac 12 \epsilon^{ijk} \phi_{ij}.
        \]
        Using this, we get:
    }
    &= \int_{\partial [0,1]^3} Q^*[\hodge \vec v].
    \intertext{%
        Then we can write out the pullback.
    }
    &= \int_{\partial [0,1]^3} [\hodge \vec v]_{ab}\del{Q(\vec x)}
    Q^a_{,i}(\vec x)
    Q^b_{,j}(\vec x)
    \dif x^i \wedge \dif x^j
    \intertext{%
        We write the component form for $\hodge \vec v$.
    }
    &= \int_{\partial [0,1]^3} \frac12 \epsilon_{abc} v^c\del{Q(\vec x)}
    Q^a_{,i}(\vec x)
    Q^b_{,j}(\vec x)
    \dif x^i \wedge \dif x^j
    \intertext{%
        The wedge product is antisymmetric in the indices $(ij)$, the Jacobians
        provide a $1:1$ mapping from $(ab)$ to $(ij)$. The Levi-Civita symbol
        antisymmetrizes in $(ab)$. Therefore we can drop the wedge product, get
        a factor of two and arrive at
    }
    &= \int_{\partial [0,1]^3} \epsilon_{abc} v^c\del{Q(\vec x)}
    Q^a_{,i}(\vec x)
    Q^b_{,j}(\vec x)
    \dif x^i \dif x^j.
    \intertext{%
        We identify the Jacobians and the coordinate surface element $\dif x^i
        \dif x^j$ with the surface element on the manifold $\dif S$. The
        Levi-Civita symbol gives a cross product and a contraction with $v$. We
        therefore have:
    }
    &= \int_{\partial [0,1]^3} \vec v\del{Q(\vec x)} \cdot \sbr{
    \sbr{
        \vec Q^a_{,i}(\vec x)
        \dif x^j
    }
    \times
    \sbr{
        \vec Q^b_{,j}(\vec x)
        \dif x^j
    }} \\
    &= \int_{\partial [0,1]^3} \vec v \cdot \dif S
\end{align*}

\subsection{Gauss' theorem}

\section{Solid angle two-form in $\R^2$}
\label{homework:2}

\section{Simplices}
\label{homework:3}

\end{document}

% vim: spell spelllang=en tw=79
