% Copyright © 2015 Martin Ueding <dev@martin-ueding.de>

\documentclass[11pt, english, fleqn, DIV=15, headinclude, BCOR=1cm]{scrartcl}

\usepackage[bibatend]{../header}

\usepackage{tikz}

\usepackage[tikz]{mdframed}
\newmdtheoremenv[%
    backgroundcolor=black!5,
    innertopmargin=\topskip,
    splittopskip=\topskip,
]{theorem}{Theorem}[section]

\usepackage{../my-boxes}

\hypersetup{
    pdftitle=
}

\newcounter{totalpoints}
\newcommand\punkte[1]{#1\addtocounter{totalpoints}{#1}}

\newcounter{problemset}
\setcounter{problemset}{6}

\subject{Geometry in Physics}
\ihead{Geometry in Physics -- Problem Set \arabic{problemset}}

\title{Problem Set \arabic{problemset}}

\publishers{Group 1 -- Jens Boos}
\ofoot{Group 1 -- Jens Boos}

\author{
    Martin Ueding \\ \small{\href{mailto:mu@martin-ueding.de}{mu@martin-ueding.de}}
    \and
    Paul Manz \\ \small{\href{mailto:paul.manz@dreiacht.de}{paul.manz@dreiacht.de}}
}
\ifoot{Martin Ueding, Paul Manz}

\ohead{\rightmark}

\usepackage{multicol}

\renewcommand\thesubsection{\thesection.\alph{subsection}}

\begin{document}

\maketitle

\vspace{3ex}

\begin{center}
    \begin{tabular}{rrr}
        problem & achieved points & possible points \\
        \midrule
        \nameref{homework:1} & & \punkte{10} \\
        \nameref{homework:2} & & \punkte{24} \\
        \nameref{homework:3} & & \punkte{16} \\
        \midrule
        Total & & \arabic{totalpoints}
    \end{tabular}
\end{center}

\section{Gauss theorem}
\label{homework:1}

\subsection{Stokes' theorem}

We start by applying the pullback.
\begin{align*}
    \int_{\partial\sigma} \phi
    &= \int_{\partial [0,1]^3} Q^* \phi
    \intertext{%
        We replace $\phi$ with $\hodge \vec v$:
        \[
            v^k := \frac 12 \epsilon^{ijk} \phi_{ij}.
        \]
        Using this, we get:
    }
    &= \int_{\partial [0,1]^3} Q^*(\hodge \vec v).
    \intertext{%
        Then we can write out the pullback.
    }
    &= \int_{\partial [0,1]^3} [\hodge \vec v]_{ab}\del{Q(\vec x)}
    Q^a_{,i}(\vec x)
    Q^b_{,j}(\vec x)
    \dif x^i \wedge \dif x^j
    \intertext{%
        We write the component form for $\hodge \vec v$.
    }
    &= \int_{\partial [0,1]^3} \frac12 \epsilon_{abc} v^c\del{Q(\vec x)}
    Q^a_{,i}(\vec x)
    Q^b_{,j}(\vec x)
    \dif x^i \wedge \dif x^j
    \intertext{%
        The wedge product is antisymmetric in the indices $(ij)$, the Jacobians
        provide a $1:1$ mapping from $(ab)$ to $(ij)$. The Levi-Civita symbol
        antisymmetrizes in $(ab)$. Therefore we can drop the wedge product, get
        a factor of two and arrive at
    }
    &= \int_{\partial [0,1]^3} \epsilon_{abc} v^c\del{Q(\vec x)}
    Q^a_{,i}(\vec x)
    Q^b_{,j}(\vec x)
    \dif x^i \dif x^j.
    \intertext{%
        We identify the Jacobians and the coordinate surface element $\dif x^i
        \dif x^j$ with the surface element on the manifold $\dif S$. The
        Levi-Civita symbol gives a cross product and a contraction with $v$. We
        therefore have:
    }
    &= \int_{\partial [0,1]^3} \vec v\del{Q(\vec x)} \cdot \sbr{
    \sbr{
        \vec Q^a_{,i}(\vec x)
        \dif x^j
    }
    \times
    \sbr{
        \vec Q^b_{,j}(\vec x)
        \dif x^j
    }} \\
    &= \int_{\partial [0,1]^3} \vec v \cdot \dif S
\end{align*}

\subsection{Gauss' theorem}

We did not get the desired result, only something in the neighborhood of it:
\begin{align*}
    \int_\sigma \dif \phi
    &= \int_\sigma Q^*(\dif \phi) \\
    &= \int_\sigma Q^*(\dif \hodge \vec v) \\
    &= \frac 12 \int_\sigma Q^*\del{\dif \del{\epsilon_{ijk} v^k \dif x^i \wedge \dif
    x^j}} \\
    &= \frac12 \int_\sigma Q^*\del{\partial_{[l} \del{\epsilon_{ij]k} v^k \dif x^i \wedge \dif
    x^j} \dif x^l} \\
    &= \int_\sigma Q^*\del{\epsilon_{ijk} v^k_{,l} \dif x^i \wedge \dif
    x^j \wedge \dif x^l} \\
    \intertext{%
        The pullback can then be applied to the free indices.
    }
    &= \int_\sigma \epsilon_{abk} v^k_{,c} \underbrace{Q^a_{,i} Q^b_{,j} Q^c_{,k} \dif x^i
    \dif x^j \dif x^k}_{\dif V}
\end{align*}
The problem is that it would need to be $\epsilon_{abc} v^l_{,l}$ to give the
right antisymmetrization and a divergence.

\section{Solid angle two-form in $\R^2$}
\label{homework:2}

\subsection{Inverse stereographic projection}

The coordinate system in $\R^3$ will be denoted by 3-vectors $\vec x$. The
coordinate system on $\R^2$ will be denoted by 2-vectors $\vec y$.

\paragraph{Inverse map}

The inverse stereographic projection from the south pole is given by:
\parencite{wikipedia/stereographische_Projektion}
\[
    h_+\inv(y_1, \ldots, y_n) = \frac{1}{|\vec y|^2 + 1} (2 y_1, \ldots, 2y_n,
    |\vec y|^2 - 1).
\]

\paragraph{Partial derivatives}

We need the partial derivatives of the map for pullback. They are:
\[
    \sbr{h_+\inv}^i_{,j} =
    \frac{2y_j}{\sbr{|\vec y|^2 + 1}^2} (2y_1, 2y_2, |\vec y|^2 - 1)^i
    + \frac{2}{|\vec y^2| + 1} \deltaup^i_j
    \eqnsep
    \sbr{h_+\inv}^{3}_{,j} =
    \frac{2y_j}{\sbr{|\vec y|^2 + 1}^2}
    .
\]
The indices $i$ and $j$ are from the index set $\{1, 2\}$.

\paragraph{Simplification of the form}

The form $\omega$ is given in Equation~(3) on the problem set. Since it is
constrained to the unit sphere $S^2$, we have
\[
    \sum_{i = 1}^3 [x^i]^2 = 1.
\]
We can therefore simplify the form to the following:
\[
    \omega_{ab} = \frac12 \epsilon_{abc} x^c.
\]
The indices $a$, $b$ and $c$ range from 1 to 3 since they describe quantities
in $\R^3$.

\paragraph{Pullback}

Now we have the needed parts to assemble the pullback of the solid angle form.
We omit some of the functional dependencies to make it fit onto the page. Every
term has to be evaluated at $(y^1, y^2)$ in some fashion.
\begin{align*}
    \sbr{\sbr{h_+\inv}^* \omega}_{ij}
    &= \omega_{ab} \del{h_+\inv(y^1, y^2)}
    \sbr{h_+\inv}^{a}_{,i}(y^1, y^2)
    \sbr{h_+\inv}^{b}_{,j}(y^1, y^2)
    \intertext{%
        We can explicitly insert $\omega$ here.
    }
    &= \frac12 \frac{1}{|\vec y|^2 + 1} \epsilon_{abc} \set{2 y^1, 2y^2,
    |\vec y|^2 -1}^c
    \sbr{h_+\inv}^{a}_{,i}
    \sbr{h_+\inv}^{b}_{,j}
    \intertext{%
        The only interesting element is $(ij) = (12)$ since it has to be
        antisymmetric and the indices only go up to 2.
    }
    \sbr{\sbr{h_+\inv}^* \omega}_{12}
    &= \frac12 \frac{1}{\sbr{[y^1]^2 + [y^2]^2} + 1} \epsilon_{abc} \set{2 y^1, 2y^2,
    |\vec y|^2 -1}^c
    \sbr{h_+\inv}^{a}_{,1}
    \sbr{h_+\inv}^{b}_{,2}
\end{align*}

Now we have to carry out the summation over $(abc)$, which is a pain. Only the
terms starting from $\epsilon$ to the right are considered. The fraction is the
exact same for all the summands, so we ignore that for now

We start with the first summand: $(abc) = (123)$.
\begin{align*}
    S_{123}
    &:= \epsilon_{123} \set{2 y^1, 2y^2, |\vec y|^2 -1}^3
    \sbr{h_+\inv}^{1}_{,1}
    \sbr{h_+\inv}^{2}_{,2} \\
    &=
    \sbr{\frac{2 y^1}{\sbr{|\vec y|^2 + 1}^2} 2y^1 + \frac{2}{|\vec y^2| + 1}}
    \sbr{\frac{2 y^2}{\sbr{|\vec y|^2 + 1}^2} 2y^2 + \frac{2}{|\vec y^2| + 1}}
    \sbr{|\vec y|^2 - 1}
    \intertext{%
        We can simplify this a tiny bit.
    }
    &=
    \frac{4 [y^1]^2 + 2 \sbr{|\vec y|^2 + 1}}{\sbr{|\vec y|^2 + 1}^2}
    \frac{4 [y^2]^2 + 2 \sbr{|\vec y|^2 + 1}}{\sbr{|\vec y|^2 + 1}^2}
    \sbr{|\vec y|^2 - 1} \\
    &=
    \frac{16 [y^1]^2 [y^2]^2 + 8 \sbr{[y^1]^2 + [y^2]^2} \sbr{|\vec y|^2 + 1}
    + \sbr{|\vec y|^2 + 1}^2}{\sbr{|\vec y|^2 + 1}^4}
    \sbr{|\vec y|^2 - 1} \\
    &=
    \sbr{
        16 \frac{[y^1]^2 [y^2]^2}{\sbr{|\vec y|^2 + 1}^4}
        + 8 \frac{|\vec y|^2}{\sbr{|\vec y|^2 + 1}^3}
        + \frac{1}{\sbr{|\vec y|^2 + 1}^2}
    }
    \sbr{|\vec y|^2 - 1}
\end{align*}

Now we go to the second summand, $(abc) = (231)$.
\begin{align*}
    S_{231}
    &:= \epsilon_{231} \set{2 y^1, 2y^2, |\vec y|^2 -1}^1
    \sbr{h_+\inv}^{2}_{,1}
    \sbr{h_+\inv}^{3}_{,2} \\
    &= 2 y^1 
    \frac{2 y^1}{\sbr{|\vec y|^2 + 1}^2} 2 y^2
    \frac{2 y^2}{\sbr{|\vec y|^2 + 1}^2} \\
    &= 16 \frac{[y^1]^2 [y^2]^2}{\sbr{|\vec y|^2 + 1}^4}
\end{align*}

The third summand, $(abc) = (312)$.
\begin{align*}
    S_{312}
    &:= \epsilon_{312} \set{2 y^1, 2y^2, |\vec y|^2 -1}^2
    \sbr{h_+\inv}^{3}_{,1}
    \sbr{h_+\inv}^{1}_{,2} \\
    &= 2 y^2
    \frac{2 y^2}{\sbr{|\vec y|^2 + 1}^2}
    \frac{2 y^1}{\sbr{|\vec y|^2 + 1}^2} 2y^1 \\
    &= 
    16 \frac{[y^1]^2 [y^2]^2}{\sbr{|\vec y|^2 + 1}^2}
\end{align*}

The fourth summand, $(abc) = (321)$.
\begin{align*}
    S_{321}
    &:= \epsilon_{321} \set{2 y^1, 2y^2, |\vec y|^2 -1}^1
    \sbr{h_+\inv}^{3}_{,1}
    \sbr{h_+\inv}^{2}_{,2} \\
    &= - 2y^1
    \frac{2 y^1}{\sbr{|\vec y|^2 + 1}^2}
    \sbr{\frac{2 y^2}{\sbr{|\vec y|^2 + 1}^2} 2y^2
    + \frac{2}{|\vec y^2| + 1}} \\
    &= - \frac{4 [y^1]^2}{\sbr{|\vec y|^2 + 1}^3}
    \sbr{\frac{4 [y^2]^2}{|\vec y|^2 + 1} + 2} \\
    &= - \frac{16 [y^1]^2 [y^2]^2}{\sbr{|\vec y|^2 + 1}^4}
    - \frac{8 [y^1]^2}{\sbr{|\vec y|^2 + 1}^3}
\end{align*}

This seems like a lot of brute force calculations. Anyway, the fifth summand,
$(abc) = (213)$.
\begin{align*}
    S_{213}
    &:= \epsilon_{213} \set{2 y^1, 2y^2, |\vec y|^2 -1}^3
    \sbr{h_+\inv}^{2}_{,1}
    \sbr{h_+\inv}^{1}_{,2} \\
    &= - \sbr{|\vec y|^2 -1}
    \frac{2 y^1}{\sbr{|\vec y|^2 + 1}^2} 2y^2
    \frac{2 y^2}{\sbr{|\vec y|^2 + 1}^2} 2y^1 \\
    &= - 16 \sbr{|\vec y|^2 -1}
    \frac{[y^1]^2 [y^2]^2}{\sbr{|\vec y|^2 + 1}^4}
\end{align*}

And finally the last one: $(abc) = (132)$.
\begin{align*}
    S_{132}
    &:= \epsilon_{132} \set{2 y^1, 2y^2, |\vec y|^2 -1}^2
    \sbr{h_+\inv}^{1}_{,1}
    \sbr{h_+\inv}^{3}_{,2} \\
    &= - 2y^2
    \sbr{\frac{2y^1}{\sbr{|\vec y|^2 + 1}^2} 2y^1 + \frac{2}{|\vec y^2| + 1}}
    \frac{2y^2}{\sbr{|\vec y|^2 + 1}^2} \\
    &= - 4
    \sbr{\frac{4 [y^1]^2}{\sbr{|\vec y|^2 + 1}^2} + \frac{2}{|\vec y^2| + 1}}
    \frac{[y^2]^2}{\sbr{|\vec y|^2 + 1}^2} \\
    &= - 16
    \frac{[y^1]^2 [y^2]^2}{\sbr{|\vec y|^2 + 1}^4}
    - 8
    \frac{[y^2]^2}{\sbr{|\vec y|^2 + 1}^3}
\end{align*}

Now we can take all those together:
\begin{align*}
    \sum S
    &=
    \sbr{
        16 \frac{[y^1]^2 [y^2]^2}{\sbr{|\vec y|^2 + 1}^4}
        + 8 \frac{|\vec y|^2}{\sbr{|\vec y|^2 + 1}^3}
        + \frac{1}{\sbr{|\vec y|^2 + 1}^2}
    }
    \sbr{|\vec y|^2 - 1}
    + 16 \frac{[y^1]^2 [y^2]^2}{\sbr{|\vec y|^2 + 1}^4}
    \\&\quad
    + 16 \frac{[y^1]^2 [y^2]^2}{\sbr{|\vec y|^2 + 1}^2}
    - \frac{16 [y^1]^2 [y^2]^2}{\sbr{|\vec y|^2 + 1}^4}
    - \frac{8 [y^1]^2}{\sbr{|\vec y|^2 + 1}^3}
    - 16 \sbr{|\vec y|^2 -1} \frac{[y^1]^2 [y^2]^2}{\sbr{|\vec y|^2 + 1}^4}
    \\&\quad
    - 16 \frac{[y^1]^2 [y^2]^2}{\sbr{|\vec y|^2 + 1}^4}
    - 8 \frac{[y^2]^2}{\sbr{|\vec y|^2 + 1}^3} \\
    \intertext{%
        We remove the last term in the first row together with part of the last
        term in the second row. We remove the first two terms of the second
        row. We combine the third term (with 8) in the second row with the last
        one on the third row. This will give an $|\vec y|^2$.
    }
    &=
    \sbr{
        16 \frac{[y^1]^2 [y^2]^2}{\sbr{|\vec y|^2 + 1}^4}
        + 8 \frac{|\vec y|^2}{\sbr{|\vec y|^2 + 1}^3}
        + \frac{1}{\sbr{|\vec y|^2 + 1}^2}
    }
    \sbr{|\vec y|^2 - 1}
    \\&\quad
    - 8 \frac{|\vec y|^2}{\sbr{|\vec y|^2 + 1}^3}
    - 16 |\vec y|^2 \frac{[y^1]^2 [y^2]^2}{\sbr{|\vec y|^2 + 1}^4}
    - 16 \frac{[y^1]^2 [y^2]^2}{\sbr{|\vec y|^2 + 1}^4}
    \intertext{%
        We expand the first line.
    }
    &=
    16 |\vec y|^2 \frac{[y^1]^2 [y^2]^2}{\sbr{|\vec y|^2 + 1}^4}
    + 8 |\vec y|^2 \frac{|\vec y|^2}{\sbr{|\vec y|^2 + 1}^3}
    + |\vec y|^2 \frac{1}{\sbr{|\vec y|^2 + 1}^2}
    - 16 \frac{[y^1]^2 [y^2]^2}{\sbr{|\vec y|^2 + 1}^4}
    - 8 \frac{|\vec y|^2}{\sbr{|\vec y|^2 + 1}^3}
    \\&\quad
    - \frac{1}{\sbr{|\vec y|^2 + 1}^2}
    - 8 \frac{|\vec y|^2}{\sbr{|\vec y|^2 + 1}^3}
    - 16 |\vec y|^2 \frac{[y^1]^2 [y^2]^2}{\sbr{|\vec y|^2 + 1}^4}
    - 16 \frac{[y^1]^2 [y^2]^2}{\sbr{|\vec y|^2 + 1}^4}
    \intertext{%
        Now we can cancel the first term in the first row with the third term
        in the second row. Then there are other terms which we can join
        together. Lastly we group them by the powers in the denominator.
    }
    &=
    - 32 \frac{[y^1]^2 [y^2]^2}{\sbr{|\vec y|^2 + 1}^4}
    + 8 \frac{|\vec y|^4 - 2 |\vec y|^2}{\sbr{|\vec y|^2 + 1}^3}
    + \frac{|\vec y|^2 - 1}{\sbr{|\vec y|^2 + 1}^2}
\end{align*}

Using this, we can write down the one interesting component of the pullback:
\begin{align*}
    \sbr{\sbr{h_+\inv}^* \omega}_{12}
    &= \frac12 \frac{1}{|\vec y|^2 + 1}
    \sbr{
        - 32 \frac{[y^1]^2 [y^2]^2}{\sbr{|\vec y|^2 + 1}^4}
        + 8 \frac{|\vec y|^4 - 2 |\vec y|^2}{\sbr{|\vec y|^2 + 1}^3}
        + \frac{|\vec y|^2 - 1}{\sbr{|\vec y|^2 + 1}^2}
    } \\
    &= 
    - 16 \frac{[y^1]^2 [y^2]^2}{\sbr{|\vec y|^2 + 1}^5}
    + 4 \frac{|\vec y|^4 - 2 |\vec y|^2}{\sbr{|\vec y|^2 + 1}^4}
    + \frac 12 \frac{|\vec y|^2 - 1}{\sbr{|\vec y|^2 + 1}^3}
\end{align*}

\subsection{Integration}

We have
\begin{align*}
    \int_S \omega
    &= \int_{\R^2} \sbr{h_+\inv}^* \omega \\
    &= 2 \int_{\R^2} \sbr{\sbr{h_+\inv}^* \omega}_{12} \dif y^1 \dif y^2
\end{align*}
We probably got the wrong result in the previous problem since we do not see
how this should give $4 \piup$, even when trying to use polar coordinates.

\subsection{Cartesian coordinates}

\subsection{Closed form}

It has to be a closed form on $S^2$ since $\omega$ is a 2-form and the space
only has two dimensions. There is no way to add another index since $\dif
\omega$ would have to be a 3-form.

\section{Simplices}
\label{homework:3}

\end{document}

% vim: spell spelllang=en tw=79
