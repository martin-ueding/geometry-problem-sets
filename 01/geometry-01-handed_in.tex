% Copyright © 2015 Martin Ueding <dev@martin-ueding.de>

\documentclass[11pt, ngerman, fleqn, DIV=15, headinclude, BCOR=1cm]{scrartcl}

\usepackage[bibatend, color]{../header}

\usepackage{tikz}

\usepackage[tikz]{mdframed}
\newmdtheoremenv[%
    backgroundcolor=black!5,
    innertopmargin=\topskip,
    splittopskip=\topskip,
]{theorem}{Theorem}[section]

\hypersetup{
    pdftitle=
}

\newcounter{totalpoints}
\newcommand\punkte[1]{#1\addtocounter{totalpoints}{#1}}

\newcounter{problemset}
\setcounter{problemset}{1}

\subject{Geometry in Physics}
\ihead{Geometry in Physics -- Problem Set \arabic{problemset}}

\title{Problem Set \arabic{problemset}}

\publishers{Group 1}
\ofoot{Group 1}

\author{
    Martin Ueding \\ \small{\href{mailto:mu@martin-ueding.de}{mu@martin-ueding.de}}
    \and
    Paul Manz \\ \small{\href{mailto:paul.manz@dreiacht.de}{paul.manz@dreiacht.de}}
}
\ifoot{Martin Ueding, Paul Manz}

\ohead{\rightmark}

\begin{document}

\maketitle

\vspace{3ex}

\begin{center}
    \begin{tabular}{rrr}
        problem number & achieved points & possible points \\
        \midrule
        1 & & \punkte{30} \\
        2 & & \punkte{20} \\
        \midrule
        Total & & \arabic{totalpoints}
    \end{tabular}
\end{center}

\vspace{5ex}

I, Martin Ueding, would like to scan and upload the problem sets with your
corrections to my website \href{http://martin-ueding.de}{martin-ueding.de}.
There, the original problem set as well as the reviewed one will be licensed
under the “\href{http://creativecommons.org/licenses/by-sa/4.0/}{Creative
Commons Attribution-ShareAlike 4.0 International License}”. Is that okay with
you?

Yes $\Box$ \hspace{2cm} No $\Box$

\newpage

\section{Differentiable Structure of $S^2$}


\section{Equivalence relations}

\subsection{Show that it is one}

\renewcommand\mod{\operatorname{mod}}

The \textbf{reflectivity} is shown quickly.
\[
    x \sim x \iff x \mod n = x \mod n
\]
is fulfilled since the equality itself obeys reflectivity.

Since the equality itself is symmetric, the \textbf{symmetry} of the
equivalence relation is shown as quick as the previous property:
\[
    x \sim y
    \iff
    x \mod n = y \mod n
    \iff
    y \mod n = x \mod n
    \iff
    y \sim x.
\]

For the \textbf{transitivity}, let $x \sim y$ and $y \sim z$. Then we use the
transitivity of the equality to show:
\begin{align*}
    x \sim y \land y \sim z
    &\iff
    x \mod n = y \mod n
    \land
    y \mod n = z \mod n \\
    &\iff
    x \mod n = y \mod n = z \mod n \\
    &\iff
    x \mod n = z \mod n \\
    &\iff
    x \sim z.
\end{align*}

\subsection{Cardinality of quotient set}

If $x \in [a]$, then by the definition of the equivalence class, we have
\[
    [a] = \set{x \in \Z \colon x \sim a}
    = \set{x \in \Z \colon x \mod n = a \mod n}.
\]
We choose the representative $a$ of the equivalence class such that
\[
    a \mod n = a
    \iff
    0 \leq a < n.
\]
Now we have
\[
    [a]
    = \set{x \in \Z \colon x \mod n = a}
    = \set{a + kn \colon k \in \Z}.
\]
The only unique $a$ are $a = 0, \ldots, n-1$ such that the number of those sets
is just $n$.

\end{document}

% vim: spell spelllang=en tw=79
