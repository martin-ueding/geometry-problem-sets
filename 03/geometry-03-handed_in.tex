% Copyright © 2015 Martin Ueding <dev@martin-ueding.de>

\documentclass[11pt, english, fleqn, DIV=15, headinclude, BCOR=1cm]{scrartcl}

\usepackage[bibatend, color]{../header}

\usepackage{tikz}

\usepackage[tikz]{mdframed}
\newmdtheoremenv[%
    backgroundcolor=black!5,
    innertopmargin=\topskip,
    splittopskip=\topskip,
]{theorem}{Theorem}[section]

\hypersetup{
    pdftitle=
}

\newcounter{totalpoints}
\newcommand\punkte[1]{#1\addtocounter{totalpoints}{#1}}

\newcounter{problemset}
\setcounter{problemset}{3}

\subject{Geometry in Physics}
\ihead{Geometry in Physics -- Problem Set \arabic{problemset}}

\title{Problem Set \arabic{problemset}}

\publishers{Group 1 -- Jens Boos}
\ofoot{Group 1 -- Jens Boos}

\author{
    Martin Ueding \\ \small{\href{mailto:mu@martin-ueding.de}{mu@martin-ueding.de}}
    \and
    Paul Manz \\ \small{\href{mailto:paul.manz@dreiacht.de}{paul.manz@dreiacht.de}}
}
\ifoot{Martin Ueding, Paul Manz}

\ohead{\rightmark}

\usepackage{multicol}

\renewcommand\thesubsection{\thesection.\alph{subsection}}

\begin{document}

\maketitle

\vspace{3ex}

\begin{center}
    \begin{tabular}{rrr}
        problem number & achieved points & possible points \\
        \midrule
        1 & & \punkte{20} \\
        2 & & \punkte{20} \\
        3 & & \punkte{10} \\
        \midrule
        Total & & \arabic{totalpoints}
    \end{tabular}
\end{center}

\section{Alternating forms}

\subsection{Vector space}

The forms we consider here are still linear by the definition on the problem
set. Since $\phi$ is linear in all of its arguments, the definition of
addition and scalar multiplication is straightforward. Since we consider the
whole space of those alternating forms, the space is also closed under
those operations. It must be a vector space.

\subsection{Small $p$}

In the definition of the space of alternating forms, the number $p$ said
how many elements of $V$ are needed for the from to produce a real value from
the field $\R$. So for $p = 0$, the resulting space is $\Lambda^0(V^*)$
which contains all the forms that do not take any element from $V$ to produce a
scalar. These are all the elements of $\R$ itself.

Now for $p = 1$ one needs to give the 1-form $\phi$ one element from $V$
to get a scalar. Since it only has one argument, it cannot be not
antisymmetric in it. So \emph{any} 1-form is antisymmetric in this sense.
Therefore all of $V^*$ is allowed and we have $\Lambda^1(V^*) = V^*$.

\subsection{Large $p$}

Here we have an $n$-dimensional vector space and consider $p$-forms with $p >
n$ on that. The problem is that there are only $n$ independent vectors in an
$n$-dimensional space. Since it is multilinear the whole thing can be expressed
in terms of the basis vectors $\{\ev_i\}$ of the vector space. The action of
$\phi$ on a set of $p$ vectors can be expanded in terms of basis vectors. We
start with expression the vectors in the basis:
\begin{align*}
    \phi(\vec v_1, \ldots, \vec v_p)
    &= \phi(v_1^i \ev_i, \ldots, v_p^k \ev_k).
    \intertext{%
        Now we can use the linearity in each argument of $\phi$ and pull all
        the scalars out.
    }
    &= v_1^{i} \ldots v_p^k \phi(\ev_i, \ldots, \ev_k).
\end{align*}
The only way a summand in the implied sum can give a non-zero contribution is
that all the indices $i, \ldots, k$ are pairwise different. Since $i, \ldots, k
\in \N \cap [1, n]$, this is not possible! The only \emph{alternating} form
that could be defined is zero.

\subsection{Dimensionality}

The dimensionality of a space is equal to the number of basis vectors. We will
show here that the number of basis vectors of the space of alternating forms is
$\binom np$. We want $p$-forms, they therefore need to take $p$ vectors as
arguments and generate a scalar from them. Since they form a vector space,
every form can be written as a linear combination of basis elements. The forms
are contained in the space $\otimes^p V$. The basis elements of this tensor
space can be
written as
\[
    \bigotimes_{k=0}^p \ev^{i_k}.
\]
where the $\{i_k\}$ are out of $\N \cap [1, n]$ but not necessarily unique at
this point. In other words, one has an $n$-dimensional space where one wants to
create a linear mapping that takes $p$ vectors as arguments. Those multilinear
mappings can be written as products of simple linear mappings. To illustrate
this with $p = 2$ here: Let $\phi(\vec u, \vec v) = \lambda$ be such a bilinear
mapping. Then we can write this as
\[
    \phi(\vec u, \vec v) = u^i v^j \phi(\ev_i, \ev_j).
\]
The $n^2$ components $\phi(\ev_i, \ev_j)$ completely describe the mapping. We
can write $\phi$ in components as
\[
    \phi = \phi_{ij} \ev^i \otimes \ev^j.
\]
Now $\phi$ can be thought of as a (0, 2) tensor which has $n^2$ components. Now
comes the restriction part: Since this tensor has to be completely
antisymmetric in all its indices, there is just one degree of freedom left, the
magnitude of the $\phi_{12}$ component. Therefore, with $n = p = 2$, it
actually is $\binom 22$. To generalize this to $n = 3$ but keeping $p = 2$ the
basis vectors are now the following:
\[
    \ev^1 \otimes \ev^2
    \eqnsep
    \ev^2 \otimes \ev^3
    \eqnsep
    \ev^3 \otimes \ev^1
\]
So there are three basis elements for two-forms in a three dimensional space.
This is because there are $\binom np$ ways of selecting $p$ unique elements of
a set with cardinality $n$. The $\phi_{ij\ldots k}$ does not introduce any more
degrees of freedom as shown above.

\end{document}

% vim: spell spelllang=en tw=79
